\documentclass[sigconf, review, anonymous]{acmart}

\usepackage{tikz}
\usetikzlibrary{math}

\usepackage{algorithm}
\usepackage{algorithmic}

\usepackage{listings}

\tikzstyle{dot}=[circle, fill=black, inner sep=0mm, minimum size=1mm]
\tikzstyle{l}=[dotted, gray, thin]
\tikzstyle{input}=[draw=none, inner sep=.2mm]
\tikzstyle{gate}=[draw, circle, inner sep=.2mm, minimum size=2mm]
\tikzstyle{outgate}         = [gate, thick]
\tikzstyle{wire}            = [draw,->]
\tikzstyle{notwire}         = [draw,->,dashed]
\tikzmath{\d=0.4;}

\usepackage[textwidth=15mm]{todonotes}

\DeclareMathOperator{\SUM}{SUM}
\DeclareMathOperator{\ADD}{ADD}
\DeclareMathOperator{\MULT}{MULT}
\DeclareMathOperator{\BA}{BA}

\begin{document}

\begin{CCSXML}
	<ccs2012>
	<concept>
	<concept_id>10010583.10010682.10010690.10010691</concept_id>
	<concept_desc>Hardware~Combinational synthesis</concept_desc>
	<concept_significance>500</concept_significance>
	</concept>
	<concept>
	<concept_id>10002950.10003624.10003625.10003628</concept_id>
	<concept_desc>Mathematics of computing~Combinatorial algorithms</concept_desc>
	<concept_significance>300</concept_significance>
	</concept>
	</ccs2012>
\end{CCSXML}

\ccsdesc[500]{Hardware~Combinational synthesis}
\ccsdesc[300]{Mathematics of computing~Combinatorial algorithms}

\keywords{multiplier, multiplication, summation, Boolean, circuit, synthesis, combinational}

\title{Smaller Circuits for Bit Addition}




\begin{abstract}
	The Full Adder and Half Adder are used frequently when synthesizing Boolean circuits.
	For example, one can compute the binary representation of~the sum of~$n$~inputs bits
	using at~most~$n$ Full and Half Adders. Also, a~circuit computing the sum of~two $n$-bit
	numbers can be~composed out of~one Half Adder and $(n-1)$ Full Adders. Finally,
	to~compute the product of~two $n$-bit numbers, one first computes $n^2$ partial products
	and then adds them using about $n^2$ Full and Half Adders. Two natural strategies for
	adding the partial products were presented by~Wallace and Dadda.

	In~this paper, we~study the size of~circuits composed out of~Full and Half Adders,
	over the full binary basis. Interestingly, many such circuits are suboptimal:
	for example, already the sum of~five input bits can be~computed more efficiently
	than by~using two Full Adders and a~single Half Adder. We~show that, in~many interesting
	regimes, circuits composed out of~Full and Half Adders, can be~made about 10\% smaller.
	This applies to~Wallace and Dadda multipliers.
	We~provide open source implementation of~generators producing efficient
	circuits for summation and multiplication.\todo{I will update the abstract later}
\end{abstract}


\maketitle

\section{Overview}
%The Full Adder and Half Adder
%(abbreviated~as FA and HA and also known~as 3:2 and 2:2 compressors, respectively)
%are used frequently when synthesizing Boolean circuits. They compute $\SUM_3$
%and $\SUM_2$, respectively, where the function
%\[\SUM_n \colon \{0,1\}^n \to \{0,1\}^l\]
%gives the binary representation of~the sum of~$n$~input bits:
%\[\SUM_n(x_1, \dotsc, x_n)=(s_0, \dotsc, s_{l-1}) \text{ such that } \sum_{i=1}^{n}x_i=\sum_{i=0}^{l-1}2^is_i,\]
%where $l=\lceil \log_2(n+1) \rceil$.

Bit addition is~one of~the most frequently used operations
in~Boolean circuit synthesis. It~arises, for example, in~the
following scenarios.
\begin{itemize}
	\item Adding two $n$-bit numbers.
	\item Computing a~symmetric Boolean function 
		(such as~majority or~sorting).
		A~natural way of~doing this is~to~first compute
		the binary representation of~the sum of~$n$~input bits
		(that~is, to~compress $n$~bits into about $\log_2 n$ bits)
		and then to~compute the function at~hand
		out of~the computed binary representation.
	\item To~multiply two $n$-bit numbers, one may first compute
		all partial products and then sum~up the resulting bits.
\end{itemize}
In~terms of~the dot-notation introduced by~Dadda~\cite{dadda}, the three scenarios discussed above are visualized as~shown in~Figure~\ref{figure:dot1}.

\begin{figure}[!ht]
	\begin{center}
	    \begin{tikzpicture}
	        %\draw[help lines] (0, -2) grid (8, 1);
	
	        \begin{scope}[xshift=25mm]
	            \foreach \y in {-2, ..., 2}
	                \node[dot] at (0, \y * \d) {};
	            \node at (0, -2) {$\SUM_5$};
	
	            \foreach \x [count=\n from 0] in {0} {
	                \draw[l] (\x * \d, -1) -- (\x * \d, 1);
	                \node[gray, below] at (\x * \d, -1) {$\n$};
	            }
	        \end{scope}
	
	        \begin{scope}[xshift=0mm]
	            \foreach \x in {-2, ..., 2} {
	                \node[dot] at (\x * \d, \d / 2) {};
	                \node[dot] at (\x * \d, - \d / 2) {};
	            }
	            \node at (0, -2) {$\ADD_5$};
	
	            \foreach \x [count=\n from 0] in {2, 1, ..., -2} {
	                \draw[l] (\x * \d, -1) -- (\x * \d, 1);
	                \node[gray, below] at (\x * \d, -1) {$\n$};
	            }
	        \end{scope}
	
	        \begin{scope}[xshift=55mm]
	            \foreach \y in {-2, ..., 2}
	                \foreach \x in {-2, ..., 2} {
	                    \node[dot] at (\x * \d + \y * \d, \y * \d) {};
	            }
	            \node at (0, -2) {$\MULT_5$};
	
	            \foreach \x [count=\n from 0] in {4, 3, ..., -4} {
	                \draw[l] (\x * \d, -1) -- (\x * \d, 1);
	                \node[gray, below] at (\x * \d, -1) {$\n$};
	            }
	        \end{scope}
	    \end{tikzpicture}
	\end{center}
	\caption{Dot diagrams for three Boolean functions: $\ADD_5$ adds two five-bit numbers, $\SUM_5$ adds five bits, and $\MULT_5$ adds five five-bit numbers.}
	\label{figure:dot1}
\end{figure}

There are many other cases where one needs to~add bits.
Say, one may want to~add a~single bit to an~$n$-bit number, 
or~to~add three $n$-bit numbers, or~add a~few bits of~varying significance,
see Figure~\ref{figure:dot2}.

\begin{figure}[!ht]
	\begin{center}
	    \begin{tikzpicture}
	        \begin{scope}
	            \foreach \x in {-2, ..., 2}
	                \node[dot] at (\x * \d, - \d / 2) {};
	            \node[dot] at (2 * \d, \d / 2) {};
	
	            \foreach \x [count=\n from 0] in {2, 1, ..., -2} {
	                \draw[l] (\x * \d, -1) -- (\x * \d, 1);
	                \node[gray, below] at (\x * \d, -1) {$\n$};
	            }
	        \end{scope}
	
	
	        \begin{scope}[xshift=30mm]
	            \foreach \x in {-2, ..., 2} {
	                \node[dot] at (\x * \d, \d) {};
	                \node[dot] at (\x * \d, 0) {};
	                \node[dot] at (\x * \d, -\d) {};
	            }
	
	            \foreach \x [count=\n from 0] in {2, 1, ..., -2} {
	                \draw[l] (\x * \d, -1) -- (\x * \d, 1);
	                \node[gray, below] at (\x * \d, -1) {$\n$};
	            }
	        \end{scope}
	
	        \begin{scope}[xshift=60mm]
	            \foreach \x/\y in {-2/-1, -1/-1, -1/0, -1/1, 1/-1, 1/0, 2/-1}
	                \node[dot] at (\x * \d, \y * \d) {};
	
	            \foreach \x [count=\n from 0] in {2, 1, ..., -2} {
	                \draw[l] (\x * \d, -1) -- (\x * \d, 1);
	                \node[gray, below] at (\x * \d, -1) {$\n$};
	            }
	        \end{scope}
	    \end{tikzpicture}
	\end{center}
	\caption{More scenarios of~adding bits of~varying significance.}
	\label{figure:dot2}
\end{figure}

A~function capturing all such scenarios is~known as~\emph{bit adder}
$\BA_{k_0, k_1, \dotsc, k_r}$. It~takes $k_0+k_1+\dotsb+k_r$ input bits:
$k_0$~bits of~significance~$0$, $k_1$~bits of~significance~$1$, and so~on.
It~outputs~$m$ bits of~significance $0,\dotsc,m-1$ where $m$~is
minimal such that $\sum_{i=0}^{r}k_i2^i \le 2^m-1$. This way, $\SUM_n=\BA_{0,0,\dotsc,0}$ and $\ADD_n=\BA_{0,0,1,1,\dotsc,n-1,n-1}$.

Since bit addition is~such a~basic task in~Boolean circuit synthesis,
a~lot of~research has been done on~constructing efficient circuits 
for various special cases of~it, see~\cite{DBLP:journals/cc/PatersonZ93}
and references therein.\todo{add more references}
A~vast majority of~these results is~devoted to~optimizing the circuit \emph{depth} (also known as~delay).
In~this paper, we~investigate the circuit \emph{size} (also known as~area) of~bit addition. Specifically, we~study circuits over the full binary basis.

Two basic building blocks for adding bits are known as~Half Adder~(HA)
and Full Adder~(FA). They compute the binary representation of~the sum
of~two and three bits, respectively (that~is, $\SUM_2$ and $\SUM_3$).
In~the full binary basis, they can be~implemented in~two and five gates, respectively, see Figure~\ref{figure:sum23}.

\begin{figure}[!ht]
	\begin{center}
		\begin{tikzpicture}[label distance=-.9mm, scale=.7]
			\foreach \n/\x/\y in {1/0/1, 2/1/1}
				\node[input] (x\n) at (\x, \y) {$x_{\n}$};
			\node[gate, label=left:$w_1$] (g1) at (0,0) {$\land$};
			\node[gate, label=right:$w_0$] (g2) at (1,0) {$\oplus$};
			\foreach \f/\t in {x1/g1, x1/g2, x2/g1, x2/g2}
				\draw[->] (\f) -- (\t);
				
			\begin{scope}[xshift=30mm, yshift=-10mm]
				\foreach \n/\x/\y in {1/0/3, 2/1/3, 3/2/3}
					\node[input] (x\n) at (\x, \y) {$x_{\n}$};
				\node[gate,label=left:$a$] (g1) at (0.5,2) {$\oplus$};
				\node[gate,label=left:$b$] (g2) at (1.5,2) {$\oplus$};
				\node[gate,label=left:$c$] (g3) at (0.5,1) {$\lor$};
				\node[gate, label=right:$w_0$] (g4) at (1.5,1) {$\oplus$};
				\node[gate, label=right:$w_1$] (g5) at (0.5,0) {$\oplus$};
				\foreach \f/\t in {x1/g1, x2/g1, x2/g2, x3/g2, g1/g3, g2/g3, g1/g4, g3/g5, g4/g5}
					\draw[->] (\f) -- (\t);
				\path (x3) edge[bend left,->] (g4);
			\end{scope}
		\end{tikzpicture}
	\end{center}
	\caption{The Half Adder and Full Adder in~the full binary basis.}
	\label{figure:sum23}
\end{figure}

Using Half Adders and Full Adders, one can synthesize a~bit adder using the following algorithm that goes back to~Napier's \emph{Rabdologiæ} (1617),
as~modernized by~Dadda~\cite{dadda}.
\begin{quote}
	Process the bits layer by~layer, in~the order of~increasing significance.
	While the current significance layer~$i$ contains at~least three bits,
	take three of~them and apply~FA to~replace them with a~pair of~bits
	of~significance $i$~and~$i+1$. If~there are two bits left at~the current layer~$i$, apply~HA to~them to~get a~pair of~bits of~significance $i$~and~$i+1$.
\end{quote}
For example, by~applying this algorithm, one can compute $\SUM_5$ using two FA's and one HA. This results in a~circuit of~size~$12$, see Figure~\ref{figure:sumfive}. Also, by~applying this algorithm to~partial products of~bits of~two input $n$-bit numbers, one gets the well-known Dadda multiplier circuit~\cite{dadda}.

\begin{figure}[!ht]
	\begin{tikzpicture}
		\begin{scope}[scale=.7]
			%\draw[help lines] (0,-5) grid (16,6);
%			\begin{scope}[yshift=-10mm]
%				\foreach \n in {1,...,5}
%				\node[input] (\n) at (\n,6) {$x_{\n}$};
%				\draw (0.5, 5.5) rectangle (3.5, 4.5); \node at (2, 5) {$\SUM_3$};
%				\foreach \n in {1, 2, 3}
%				\draw[->] (\n) -- (\n, 5.5);
%				\draw (2.5, 3.5) rectangle (5.5, 2.5); \node at (4, 3) {$\SUM_3$};
%				\path (3, 4.5) edge[->] node[l] {0} (3, 3.5);
%				\foreach \n in {4, 5}
%				\draw[->] (\n) -- (\n, 3.5);
%				\draw (1.5, 1.5) rectangle (3.5, 0.5); \node at (2.5, 1) {$\SUM_2$};
%				\path (2, 4.5) edge[->] node[l] {1} (2, 1.5);
%				\path (3, 2.5) edge[->] node[l] {1} (3, 1.5);
%				\node[input] (w2) at (2,-1) {$w_2$};
%				\node[input] (w1) at (3,-1) {$w_1$};
%				\node[input] (w0) at (4.5,-1) {$w_0$};
%				\path (2, 0.5) edge[->] node[l] {1} (w2);
%				\path (3, 0.5) edge[->] node[l] {0} (w1);
%				\path (4.5, 2.5) edge[->] node[l] {0} (w0);
%			\end{scope}
			
			\begin{scope}[label distance=-1mm, xshift=70mm, yshift=20mm]
				\foreach \n/\x/\y in {1/0/3, 2/1/3, 3/2/3, 4/2.5/1, 5/3.5/1}
				\node[input] (x\n) at (\x, \y) {$x_{\n}$};
				\node[gate,label=left:$g_1$] (g1) at (0.5,2) {$\oplus$};
				\node[gate,label=left:$g_2$] (g2) at (1.5,2) {$\oplus$};
				\node[gate,label=left:$g_3$] (g3) at (0.5,1) {$\lor$};
				\node[gate,label=left:$g_4$] (g4) at (1.5,1) {$\oplus$};
				\node[gate,label=left:$g_5$] (g5) at (0.5,0) {$\oplus$};
				\node[gate,label=left:$g_6$] (g6) at (2,-1) {$\oplus$};
				\node[gate,label=right:$g_7$] (g7) at (3,-1) {$\oplus$};
				\node[gate,label=right:$g_8$] (g8) at (2,-2) {$\lor$};
				\node[gate, label=right:$w_0$] (g9) at (3,-2) {$\oplus$};
				\node[gate, label=right:$g_9$] (g10) at (2,-3) {$\oplus$};
				\node[gate, label=right:$w_1$] (g11) at (2,-4) {$\oplus$};
				\node[gate, label=left:$w_2$] (g12) at (1,-4) {$\land$};
				
				\foreach \f/\t in {x1/g1, x2/g1, x2/g2, x3/g2, g1/g3, g2/g3, g1/g4, g3/g5, g4/g5, g4/g6, x4/g6, x4/g7, x5/g7, g6/g8, g7/g8, g8/g10, g6/g9, g9/g10, g10/g11, g10/g12}
				\draw[->] (\f) -- (\t);
				
				\path (x3) edge[->,bend left] (g4);
				\path (x5) edge[->,bend left=35] (g9);
				\path (g5) edge[->,bend right=25] (g11);
				\path (g5) edge[->,bend right=15] (g12);
				
				\draw[dashed] (-0.5,-0.25) rectangle (2,2.5);
				\draw[dashed] (1.25,-3.25) rectangle (4,-0.5);
				\draw[dashed] (0,-3.5) rectangle (3,-4.5);
			\end{scope}
		\end{scope}
	\end{tikzpicture}
	\caption{A~circuit of~size~$12$ computing~$\SUM_5$ composed of~two Full Adders and one Half Adder.}
	\label{figure:sumfive}
\end{figure}

Interestingly, in~some regimes, this algorithm leads to~a~circuit 
that is~\emph{provably} optimal. For example, it~adds two $n$-bit 
numbers using a~single Half Adder and $n-1$ Full Adders. The resulting
circuit is~known as~\emph{ripple-carry adder} and has size $2+5(n-1)=5n-3$. Red'kin~\cite{Red81} proved
that there~is no~smaller circuit for this function.
At~the same time, in~various other scenarios, the algorithm leads 
to~a~suboptimal circuit. For example, the circuit from Figure~\ref{figure:sumfive} is~suboptimal as~$\SUM_5$
can be~computed by a~circuit of~size~$11$ (see~\cite{DBLP:conf/mfcs/KulikovPS22}).
In~general, whereas the algorithm produces a~circuit of~size about~$5n$
for $\SUM_5$, this function can be~computed by~a~circuit of~size about $4.5n$
as~shown by~Demenkov et~al.~\cite{DBLP:journals/ipl/DemenkovKKY10}.

In~this paper, we~generalize the construction by~Demenkov et~al.
Namely, we~show that whereas the algorithm described above produces
a~circuit of~size at~most $5n-3m$ computing $\BA$, one can construct
a~circuit of~size at~most $4.5n-m$. In~the regimes where $m$~is~small
compared to~$n$, this gives a~circuit that is~about $10\%$ smaller.
This applies to~the Dadda multiplier.
We~complement our theoretical result by~an~open source implementation
of~generators producing circuits for bit addition and multiplication.

\section{General Setting}
For a~predicate~$P$, $[P]$ is~the Iverson bracket: $[P]=1$ if $P$~is true and $[P]=0$ otherwise. For a~non-negative integer~$q$,
$\operatorname{bin}(q)$ is~the binary representation of~$q$
(padded with a~number of~leading zeroes if~needed).
Conversely, for a~bit-string $b=(b_0,\dotsc, b_k)$, $\operatorname{int}(b)=\sum_{i=0}^{k}2^ib_i$ is~the corresponding integer.

\subsection{Boolean Functions}
Let $B_{n,m}=\{f \colon \{0,1\}^n \to \{0,1\}^m\}$ be~the set of~all Boolean functions with $n$~inputs and $m$~outputs
and let $B_n=B_{n,1}$ be the set of~all $n$-input single-output functions
(that is, predicates).
A~function of the form $f \colon \{0,1\}^n \to \{0,1,*\}^m$
is~called \emph{partially defined}: $*$ is~known~as \emph{don't care} symbol
and means an~undefined Boolean value.

Below, we~define a~number of~specific Boolean functions studied in~this paper.
By $x=(x_1,\dotsc, x_n)$~and~$y=(y_1, \dotsc, y_n)$ we~denote input $n$-bit strings
and $\operatorname{sum}(x)=x_1+\dotsb+x_n$.
\begin{itemize}
	\item $\operatorname{MAJ}_n \in B_n$ is~the majority function, that is, it~is equal to~$1$ if~and only~if more than half of~the $n$~input bits are~$1$'s:
	\(\operatorname{MAJ}_n(x)=[\operatorname{sum}(x) > n/2].\)
	\item $\operatorname{SUM}_n \in B_{n, \lceil \log_2(n+1) \rceil}$ computes the binary representation of~the sum of~$n$~input bits:
	\(\operatorname{SUM}_n(x)=\operatorname{bin}(\operatorname{sum}(x)).\)
	\item $\operatorname{SORT}_n \in B_{n,n}$ sorts the given $n$~bits:
	\(\operatorname{SORT}_n(x)=(x_1', \dotsc, x_n'),\)
	where $x_1' \le \dotsb \le x_n'$ and $\operatorname{sum}(x)=\operatorname{sum}(x')$.
	\item $\operatorname{MULT}_n \in B_{2n, 2n}$ computes the product of~the given two~$n$-bit integers:
	\(\operatorname{MULT}_n(x, y)=\operatorname{bin}(\operatorname{int}(x) \cdot \operatorname{int}(y)).\)
	\item $\operatorname{SQR}_n \in B_{n, 2n}$ computes the square of~the given $n$-bit integer:
	\(\operatorname{SQR}_n(x)=\operatorname{MULT}_n(x, x).\)
	\item $\operatorname{SQRT}_n \in B_{n, n/2}$ computes the square root of~the given $n$-bit integer:
	\(\operatorname{SQRT}_n(x)=\operatorname{bin}(\lfloor \sqrt{\operatorname{int}(x)}\rfloor).\)
	\item $\operatorname{DIV}_n \in B_{2n, n}$ and $\operatorname{MOD}_n \in B_{2n, n}$
	functions compute, respectively, the quotient and the remainder of~the first input integer divided by~the second input integer:
	\begin{align*}
		\operatorname{DIV}_n(x,y)&=\operatorname{bin}(\lfloor \operatorname{int}(x) /\operatorname{int(y)}\rfloor),\\
		\operatorname{MOD}_n(x,y)&=\operatorname{bin}(\operatorname{int}(x) \bmod \operatorname{int(y)}).
	\end{align*}
\end{itemize}

\subsection{Boolean Circuits}
A~circuit is~a~natural way of~computing Boolean functions.
It~is an~acyclic directed graph of in-degree at~most~$2$ whose $n$~source
nodes are labeled with input variables
$x_1, \dotsc, x_n$ and all other nodes (called \emph{internal})
are labeled with Boolean operations from
$B_1 \cup B_2$ (that~is, unary and binary Boolean predicates).
The nodes of~the circuit are called \emph{gates} and each gate computes
a~(single-output) Boolean function of~$x_1, \dotsc, x_n$. Thus, if~$m$ gates of the
circuit are marked as~outputs, it~computes a~function from $B_{n,m}$.
The size of a~circuit is~its number of~internal binary gates
(it~is common to~assume that unary gates are given for free).

Figure~\ref{fig:fulladder} shows an~example of~a~circuit of~size~$5$ computing
$\operatorname{SUM}_3$. It~also highlights that a~circuit corresponds
to~an~extremely simple program (called a~straight line program):
every line of~this program just applies a~unary or~binary Boolean operation
to~input bits or~the results of~the previous lines.


\begin{figure}[tb]
	\begin{center}
		\begin{tikzpicture}[label distance=-.9mm, yscale=1]
			%\draw[help lines] (0, 0) grid (6, 6);
			\foreach \n/\x/\y in {1/0/3, 2/1/3, 3/2/3}
			\node[input] (x\n) at (\x, \y) {$x_{\n}$};
			\node[gate, label=left:$a$] (g1) at (0.5,2) {$\oplus$};
			\node[gate, label=left:$b$] (g2) at (1.5,2) {$\oplus$};
			\node[gate, label=left:$c$] (g3) at (0.5,1) {$\lor$};
			\node[outgate, label=right:$w_0$] (g4) at (1.5,1) {$\oplus$};
			\node[outgate, label=right:$w_1$] (g5) at (0.5,0) {$\oplus$};
			\foreach \f/\t in {x1/g1, x2/g1, x2/g2, x3/g2, g1/g3, g2/g3, g1/g4, g3/g5, g4/g5}
			\draw[wire] (\f) -- (\t);
			\path (x3) edge[bend left, wire] (g4);
			
			\node (c) at (0.5, -1) {\strut carry};
			\node (s) at (1.5, -1) {\strut sum};
			\draw[wire] (g4) -- (s);
			\draw[wire] (g5) -- (c);
			
			\node[right] at (3, 1) {
				\begin{lstlisting}[language=Python]
					def sum3(x1, x2, x3):
					a = x1 ^ x2
					b = x2 ^ x3
					c = a | b
					w0 = a ^ x3
					w1 = c ^ w0
					return w0, w1
				\end{lstlisting}
			};
		\end{tikzpicture}
	\end{center}
	\caption{A~circuit (known as~Full Adder) and the corresponding straight line program
		(in~\texttt{Python}) for $\operatorname{SUM}_3$. The output gates are shown in~bold.}
	\label{fig:fulladder}
\end{figure}

We~assume that a~gate of~a~circuit can compute any unary or~binary Boolean function.
It~is not difficult to~see that, for a~given Boolean function~$f$,
the minimum size of a~circuit computing~$f$ is~equal
to~the minimum size of a~circuit for~$f$
when each gate computes either
binary XOR ($\oplus$), binary AND ($\land$), or unary NOT ($\neg$):
indeed, every binary Boolean operation is~either a~summation
or a~multiplication with possibly negated inputs and outputs. For this reason,
when all unary and binary operations are allowed, we~say that this is an~XAIG circuit:
X~stands for XOR (summation),
A~stands for AND (multiplication),
I~stands for inverter (negation),
and G~stands for a~graph.

It~is~well known that any Boolean function can also be~computed
by~a~circuit that only uses AND's and NOT's as~operations in~gates.
They are called AIG circuits \cite{BiereHeljankoWieringa2011} and this is a~convenient format for representing
a~circuit in~practice: since every binary gate computes an~AND, one just stores
a~graph of~in-degree~$2$ (without storing the operations computed in~the gates)
and a~list of~flags telling which of~the edges are negated (or~inverted).
See an~example in~Figure~\ref{fig:fulladderaig}.

\begin{figure}[tb]
	\begin{center}
		\begin{tikzpicture}[yscale=1]
			%\draw[help lines] (0, 0) grid (6, 6);
			\foreach \n/\x/\y in {1/0/1, 2/1/3, 3/2/3}
			\node[input] (x\n) at (\x, \y) {$x_{\n}$};
			\foreach \n/\x/\y/\op/\p/\q in {1/2/2/\land/x2/x3, 2/1/2/\lor/x2/x3, 3/1/1/>/1/2, 4/0/0/\lor/x1/3, 5/1/0/\land/x1/3} {
				\node[gate] (\n) at (\x, \y) {$\op$};
				\draw[wire] (\p) -- (\n);
				\draw[wire] (\q) -- (\n);
			}
			\foreach \n/\x/\y/\op/\p/\q in {6/0/-1/>/4/5, 7/2/-1/\lor/1/5} {
				\node[outgate] (\n) at (\x, \y) {$\op$};
				\draw[wire] (\p) -- (\n);
				\draw[wire] (\q) -- (\n);
			}
			%\node at (1, -2.5) {(a)};
			
			\node (c) at (2, -2) {\strut carry};
			\node (s) at (0, -2) {\strut sum};
			\draw[wire] (6) -- (s);
			\draw[wire] (7) -- (c);
			
			\begin{scope}[xshift=40mm]
				\foreach \n/\x/\y in {1/0/1, 2/1/3, 3/2/3}
				\node[input] (x\n) at (\x, \y) {$x_{\n}$};
				\foreach \n/\x/\y/\op/\p/\wp/\q/\wq in {
					1/2/2/\land/x2/wire/x3/wire,
					2/1/2/\lor/x2/notwire/x3/notwire,
					3/1/1/>/1/notwire/2/notwire,
					4/0/0/\lor/x1/notwire/3/notwire,
					5/1/0/\land/x1/wire/3/wire}
				{
					\node[gate] (\n) at (\x, \y) {};
					\draw[\wp] (\p) -- (\n);
					\draw[\wq] (\q) -- (\n);
				}
				\foreach \n/\x/\y/\op/\p/\wp/\q/\wq in {
					6/0/-1/>/4/notwire/5/notwire,
					7/2/-1/\lor/1/notwire/5/notwire}
				{
					\node[outgate] (\n) at (\x, \y) {};
					\draw[\wp] (\p) -- (\n);
					\draw[\wq] (\q) -- (\n);
				}
				%\node at (1, -2.5) {(b)};
				
				\node (c) at (2, -2) {\strut carry};
				\node (s) at (0, -2) {\strut sum};
				\draw[wire] (6) -- (s);
				\draw[notwire] (7) -- (c);
			\end{scope}
		\end{tikzpicture}
	\end{center}
	\caption{A~circuit over the basis $B_2 \setminus \{\oplus, \equiv\}$ computing
		$\operatorname{SUM}_3$ (left) and its AIG representation (right). The output gates are shown in~bold, whereas the negated wires are~shown dashed. The binary Boolean operation~$>$
		is~defined in a~natural way: $a>b=a \land \overline{b}$.}
	\label{fig:fulladderaig}
\end{figure}

\begin{figure*}%[!ht]
	\begin{center}
		\begin{tikzpicture}[xscale=0.75, yscale=.6]
			\foreach \n in {2,...,17}
			\node[input] (\n) at (\n,6) {$x_{\n}$};
			\foreach \i in {1,...,8} {
				\tikzmath{\k=int(2*\i); \j=int(2*\i+1);}
				\node[gate] (a) at (\j,5) {$\oplus$};
				\draw[->] (\k) -- (a); \draw[->] (\j) -- (a);
				\draw[->] (a) -- (\j,4); \draw[->] (\k) -- (\k,4);
			}
			\foreach \x/\y/\w in {2/4/3, 6/4/3, 10/4/3, 14/4/3, 2/2.5/7, 10/2.5/7, 2/1/15} {
				\draw (\x-0.15,\y) rectangle (\x+\w+0.15,\y-1);
				\node at (\x+\w/2,\y-0.5) {MDFA};
			}
			\foreach \y/\l/\w in {3.5/x_1/w_0, 2/0/w_1, 0.5/0/w_2} {
				\node[input] (b) at (1,\y) {$\l$};
				\draw[->] (b) -- (1.85,\y);
				\node[input] (c) at (18,\y) {$\w$};
				\draw[->] (17.15,\y) -- (c);
			}
			\foreach \x in {3, 4, 7, 8, 11, 12, 15, 16}
			\draw[->] (\x,3) -- (\x,2.5);
			\foreach \x in {4, 7, 12, 15}
			\draw[->] (\x,1.5) -- (\x,1);
			\foreach \x/\y in {5/3.5, 9/3.5, 13/3.5, 9/2}
			\draw[->] (\x+0.15,\y) -- (\x+0.85,\y);
			
			\node[input] (w3) at (12,-2) {$w_3$};
			\draw[->] (12,0) -- (w3);
			\node[gate] (x) at (7,-1) {$>$};
			\draw[->] (7,0) -- (x);
			\node[input] (w4) at (7,-2) {$w_4$};
			\draw[->] (x) -- (w4);
			\path (12,0) edge[->,out=-135,in=0] (x);
		\end{tikzpicture}
	\end{center}
	\caption{A~circuit computing $\SUM_{17}$ composed out of~MDFA blocks.}
	\label{figure:sum17}
\end{figure*}



\bibliographystyle{ACM-Reference-Format}
\bibliography{references}

\end{document}

