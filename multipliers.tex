\documentclass[sigconf, review, anonymous]{acmart}

\usepackage{tikz}
\usetikzlibrary{math}

\tikzstyle{dot}=[circle, fill=black, inner sep=0mm, minimum size=1mm]
\tikzstyle{l}=[dotted, gray, thin]
\tikzstyle{input}=[draw=none, inner sep=.2mm]
\tikzstyle{gate}=[draw, circle, inner sep=.2mm, minimum size=2mm]
%\tikzstyle{l}=[draw=none, rectangle, fill=white, inner sep=.4mm]
\tikzmath{\d=0.4;}

\usepackage[textwidth=15mm]{todonotes}

\DeclareMathOperator{\SUM}{SUM}
\DeclareMathOperator{\ADD}{ADD}
\DeclareMathOperator{\MULT}{MULT}
\DeclareMathOperator{\BA}{BA}

\begin{document}

\begin{CCSXML}
	<ccs2012>
	<concept>
	<concept_id>10010583.10010682.10010690.10010691</concept_id>
	<concept_desc>Hardware~Combinational synthesis</concept_desc>
	<concept_significance>500</concept_significance>
	</concept>
	<concept>
	<concept_id>10002950.10003624.10003625.10003628</concept_id>
	<concept_desc>Mathematics of computing~Combinatorial algorithms</concept_desc>
	<concept_significance>300</concept_significance>
	</concept>
	</ccs2012>
\end{CCSXML}

\ccsdesc[500]{Hardware~Combinational synthesis}
\ccsdesc[300]{Mathematics of computing~Combinatorial algorithms}

\keywords{multiplier, multiplication, summation, Boolean, circuit, synthesis, combinational}

\title{Smaller Circuits for Bit Addition}




\begin{abstract}
	The Full Adder and Half Adder are used frequently when synthesizing Boolean circuits.
	For example, one can compute the binary representation of~the sum of~$n$~inputs bits
	using at~most~$n$ Full and Half Adders. Also, a~circuit computing the sum of~two $n$-bit
	numbers can be~composed out of~one Half Adder and $(n-1)$ Full Adders. Finally,
	to~compute the product of~two $n$-bit numbers, one first computes $n^2$ partial products
	and then adds them using about $n^2$ Full and Half Adders. Two natural strategies for
	adding the partial products were presented by~Wallace and Dadda.

	In~this paper, we~study the size of~circuits composed out of~Full and Half Adders,
	over the full binary basis. Interestingly, many such circuits are suboptimal:
	for example, already the sum of~five input bits can be~computed more efficiently
	than by~using two Full Adders and a~single Half Adder. We~show that, in~many interesting
	regimes, circuits composed out of~Full and Half Adders, can be~made about 10\% smaller.
	This applies to~Wallace and Dadda multipliers.
	We~provide open source implementation of~generators producing efficient
	circuits for summation and multiplication.\todo{I will update the abstract later}
\end{abstract}


\maketitle

\section{Overview}
%The Full Adder and Half Adder
%(abbreviated~as FA and HA and also known~as 3:2 and 2:2 compressors, respectively)
%are used frequently when synthesizing Boolean circuits. They compute $\SUM_3$
%and $\SUM_2$, respectively, where the function
%\[\SUM_n \colon \{0,1\}^n \to \{0,1\}^l\]
%gives the binary representation of~the sum of~$n$~input bits:
%\[\SUM_n(x_1, \dotsc, x_n)=(s_0, \dotsc, s_{l-1}) \text{ such that } \sum_{i=1}^{n}x_i=\sum_{i=0}^{l-1}2^is_i,\]
%where $l=\lceil \log_2(n+1) \rceil$.

Bit addition is~one of~the most frequently used operations
in~Boolean circuit synthesis. It~arises, for example, in~the
following scenarios.
\begin{itemize}
	\item Adding two $n$-bit numbers.
	\item Computing a~symmetric Boolean function 
		(such as~majority or~sorting).
		A~natural way of~doing this is~to~first compute
		the binary representation of~the sum of~$n$~input bits
		(that~is, to~compress $n$~bits into about $\log_2 n$ bits)
		and then to~compute the function at~hand
		out of~the computed binary representation.
	\item To~multiply two $n$-bit numbers, one may first compute
		all partial products and then sum~up the resulting bits.
\end{itemize}
In~terms of~the dot-notation introduced by~Dadda~\cite{dadda}, the three scenarios discussed above are visualized as~shown in~Figure~\ref{figure:dot1}.

\begin{figure}[!ht]
	\begin{center}
	    \begin{tikzpicture}
	        %\draw[help lines] (0, -2) grid (8, 1);
	
	        \begin{scope}[xshift=25mm]
	            \foreach \y in {-2, ..., 2}
	                \node[dot] at (0, \y * \d) {};
	            \node at (0, -2) {$\SUM_5$};
	
	            \foreach \x [count=\n from 0] in {0} {
	                \draw[l] (\x * \d, -1) -- (\x * \d, 1);
	                \node[gray, below] at (\x * \d, -1) {$\n$};
	            }
	        \end{scope}
	
	        \begin{scope}[xshift=0mm]
	            \foreach \x in {-2, ..., 2} {
	                \node[dot] at (\x * \d, \d / 2) {};
	                \node[dot] at (\x * \d, - \d / 2) {};
	            }
	            \node at (0, -2) {$\ADD_5$};
	
	            \foreach \x [count=\n from 0] in {2, 1, ..., -2} {
	                \draw[l] (\x * \d, -1) -- (\x * \d, 1);
	                \node[gray, below] at (\x * \d, -1) {$\n$};
	            }
	        \end{scope}
	
	        \begin{scope}[xshift=55mm]
	            \foreach \y in {-2, ..., 2}
	                \foreach \x in {-2, ..., 2} {
	                    \node[dot] at (\x * \d + \y * \d, \y * \d) {};
	            }
	            \node at (0, -2) {$\MULT_5$};
	
	            \foreach \x [count=\n from 0] in {4, 3, ..., -4} {
	                \draw[l] (\x * \d, -1) -- (\x * \d, 1);
	                \node[gray, below] at (\x * \d, -1) {$\n$};
	            }
	        \end{scope}
	    \end{tikzpicture}
	\end{center}
	\caption{Dot diagrams for three Boolean functions: $\ADD_5$ adds two five-bit numbers, $\SUM_5$ adds five bits, and $\MULT_5$ adds five five-bit numbers.}
	\label{figure:dot1}
\end{figure}

There are many other cases where one needs to~add bits.
Say, one may want to~add a~single bit to an~$n$-bit number, 
or~to~add three $n$-bit numbers, or~add a~few bits of~varying significance,
see Figure~\ref{figure:dot2}.

\begin{figure}[!ht]
	\begin{center}
	    \begin{tikzpicture}
	        \begin{scope}
	            \foreach \x in {-2, ..., 2}
	                \node[dot] at (\x * \d, - \d / 2) {};
	            \node[dot] at (2 * \d, \d / 2) {};
	
	            \foreach \x [count=\n from 0] in {2, 1, ..., -2} {
	                \draw[l] (\x * \d, -1) -- (\x * \d, 1);
	                \node[gray, below] at (\x * \d, -1) {$\n$};
	            }
	        \end{scope}
	
	
	        \begin{scope}[xshift=30mm]
	            \foreach \x in {-2, ..., 2} {
	                \node[dot] at (\x * \d, \d) {};
	                \node[dot] at (\x * \d, 0) {};
	                \node[dot] at (\x * \d, -\d) {};
	            }
	
	            \foreach \x [count=\n from 0] in {2, 1, ..., -2} {
	                \draw[l] (\x * \d, -1) -- (\x * \d, 1);
	                \node[gray, below] at (\x * \d, -1) {$\n$};
	            }
	        \end{scope}
	
	        \begin{scope}[xshift=60mm]
	            \foreach \x/\y in {-2/-1, -1/-1, -1/0, -1/1, 1/-1, 1/0, 2/-1}
	                \node[dot] at (\x * \d, \y * \d) {};
	
	            \foreach \x [count=\n from 0] in {2, 1, ..., -2} {
	                \draw[l] (\x * \d, -1) -- (\x * \d, 1);
	                \node[gray, below] at (\x * \d, -1) {$\n$};
	            }
	        \end{scope}
	    \end{tikzpicture}
	\end{center}
	\caption{More scenarios of~adding bits of~varying significance.}
	\label{figure:dot2}
\end{figure}

A~function capturing all such scenarios is~known as~\emph{bit adder}
$\BA_{k_0, k_1, \dotsc, k_r}$. It~takes $k_0+k_1+\dotsb+k_r$ input bits:
$k_0$~bits of~significance~$0$, $k_1$~bits of~significance~$1$, and so~on.
It~outputs~$m$ bits of~significance $0,\dotsc,m-1$ where $m$~is
minimal such that $\sum_{i=0}^{r}k_i2^i \le 2^m-1$. This way, $\SUM_n=\BA_{0,0,\dotsc,0}$ and $\ADD_n=\BA_{0,0,1,1,\dotsc,n-1,n-1}$.

Since bit addition is~such a~basic task in~Boolean circuit synthesis,
a~lot of~research has been done on~constructing efficient circuits 
for various special cases of~it, see~\cite{DBLP:journals/cc/PatersonZ93}
and references therein.\todo{add more references}
A~vast majority of~these results is~devoted to~optimizing the circuit \emph{depth} (also known as~delay).
In~this paper, we~investigate the circuit \emph{size} (also known as~area) of~bit addition. Specifically, we~study circuits over the full binary basis.

Two basic building blocks for adding bits are known as~Half Adder~(HA)
and Full Adder~(FA). They compute the binary representation of~the sum
of~two and three bits, respectively (that~is, $\SUM_2$ and $\SUM_3$).
In~the full binary basis, they can be~implemented in~two and five gates, respectively, see Figure~\ref{figure:sum23}.

\begin{figure}[!ht]
	\begin{center}
		\begin{tikzpicture}[label distance=-.9mm, scale=.7]
			\foreach \n/\x/\y in {1/0/1, 2/1/1}
				\node[input] (x\n) at (\x, \y) {$x_{\n}$};
			\node[gate, label=left:$w_1$] (g1) at (0,0) {$\land$};
			\node[gate, label=right:$w_0$] (g2) at (1,0) {$\oplus$};
			\foreach \f/\t in {x1/g1, x1/g2, x2/g1, x2/g2}
				\draw[->] (\f) -- (\t);
				
			\begin{scope}[xshift=30mm, yshift=-10mm]
				\foreach \n/\x/\y in {1/0/3, 2/1/3, 3/2/3}
					\node[input] (x\n) at (\x, \y) {$x_{\n}$};
				\node[gate,label=left:$a$] (g1) at (0.5,2) {$\oplus$};
				\node[gate,label=left:$b$] (g2) at (1.5,2) {$\oplus$};
				\node[gate,label=left:$c$] (g3) at (0.5,1) {$\lor$};
				\node[gate, label=right:$w_0$] (g4) at (1.5,1) {$\oplus$};
				\node[gate, label=right:$w_1$] (g5) at (0.5,0) {$\oplus$};
				\foreach \f/\t in {x1/g1, x2/g1, x2/g2, x3/g2, g1/g3, g2/g3, g1/g4, g3/g5, g4/g5}
					\draw[->] (\f) -- (\t);
				\path (x3) edge[bend left,->] (g4);
			\end{scope}
		\end{tikzpicture}
	\end{center}
	\caption{The Half Adder and Full Adder in~the full binary basis.}
	\label{figure:sum23}
\end{figure}

Using Half Adders and Full Adders, one can synthesize a~bit adder using the following algorithm that goes back to~Napier's \emph{Rabdologiæ} (1617),
as~modernized by~Dadda~\cite{dadda}.
\begin{quote}
	Process the bits layer by~layer, in~the order of~increasing significance.
	While the current significance layer~$i$ contains at~least three bits,
	take three of~them and apply~FA to~replace them with a~pair of~bits
	of~significance $i$~and~$i+1$. If~there are two bits left at~the current layer~$i$, apply~HA to~them to~get a~pair of~bits of~significance $i$~and~$i+1$.
\end{quote}
For example, by~applying this algorithm, one can compute $\SUM_5$ using two FA's and one HA. This results in a~circuit of~size~$12$, see Figure~\ref{figure:sumfive}. Also, by~applying this algorithm to~partial products of~bits of~two input $n$-bit numbers, one gets the well-known Dadda multiplier circuit~\cite{dadda}.

\begin{figure}[!ht]
	\begin{tikzpicture}
		\begin{scope}[scale=.7]
			%\draw[help lines] (0,-5) grid (16,6);
%			\begin{scope}[yshift=-10mm]
%				\foreach \n in {1,...,5}
%				\node[input] (\n) at (\n,6) {$x_{\n}$};
%				\draw (0.5, 5.5) rectangle (3.5, 4.5); \node at (2, 5) {$\SUM_3$};
%				\foreach \n in {1, 2, 3}
%				\draw[->] (\n) -- (\n, 5.5);
%				\draw (2.5, 3.5) rectangle (5.5, 2.5); \node at (4, 3) {$\SUM_3$};
%				\path (3, 4.5) edge[->] node[l] {0} (3, 3.5);
%				\foreach \n in {4, 5}
%				\draw[->] (\n) -- (\n, 3.5);
%				\draw (1.5, 1.5) rectangle (3.5, 0.5); \node at (2.5, 1) {$\SUM_2$};
%				\path (2, 4.5) edge[->] node[l] {1} (2, 1.5);
%				\path (3, 2.5) edge[->] node[l] {1} (3, 1.5);
%				\node[input] (w2) at (2,-1) {$w_2$};
%				\node[input] (w1) at (3,-1) {$w_1$};
%				\node[input] (w0) at (4.5,-1) {$w_0$};
%				\path (2, 0.5) edge[->] node[l] {1} (w2);
%				\path (3, 0.5) edge[->] node[l] {0} (w1);
%				\path (4.5, 2.5) edge[->] node[l] {0} (w0);
%			\end{scope}
			
			\begin{scope}[label distance=-1mm, xshift=70mm, yshift=20mm]
				\foreach \n/\x/\y in {1/0/3, 2/1/3, 3/2/3, 4/2.5/1, 5/3.5/1}
				\node[input] (x\n) at (\x, \y) {$x_{\n}$};
				\node[gate,label=left:$g_1$] (g1) at (0.5,2) {$\oplus$};
				\node[gate,label=left:$g_2$] (g2) at (1.5,2) {$\oplus$};
				\node[gate,label=left:$g_3$] (g3) at (0.5,1) {$\lor$};
				\node[gate,label=left:$g_4$] (g4) at (1.5,1) {$\oplus$};
				\node[gate,label=left:$g_5$] (g5) at (0.5,0) {$\oplus$};
				\node[gate,label=left:$g_6$] (g6) at (2,-1) {$\oplus$};
				\node[gate,label=right:$g_7$] (g7) at (3,-1) {$\oplus$};
				\node[gate,label=right:$g_8$] (g8) at (2,-2) {$\lor$};
				\node[gate, label=right:$w_0$] (g9) at (3,-2) {$\oplus$};
				\node[gate, label=right:$g_9$] (g10) at (2,-3) {$\oplus$};
				\node[gate, label=right:$w_1$] (g11) at (2,-4) {$\oplus$};
				\node[gate, label=left:$w_2$] (g12) at (1,-4) {$\land$};
				
				\foreach \f/\t in {x1/g1, x2/g1, x2/g2, x3/g2, g1/g3, g2/g3, g1/g4, g3/g5, g4/g5, g4/g6, x4/g6, x4/g7, x5/g7, g6/g8, g7/g8, g8/g10, g6/g9, g9/g10, g10/g11, g10/g12}
				\draw[->] (\f) -- (\t);
				
				\path (x3) edge[->,bend left] (g4);
				\path (x5) edge[->,bend left=35] (g9);
				\path (g5) edge[->,bend right=25] (g11);
				\path (g5) edge[->,bend right=15] (g12);
				
				\draw[dashed] (-0.5,-0.25) rectangle (2,2.5);
				\draw[dashed] (1.25,-3.25) rectangle (4,-0.5);
				\draw[dashed] (0,-3.5) rectangle (3,-4.5);
			\end{scope}
		\end{scope}
	\end{tikzpicture}
	\caption{A~circuit of~size~$12$ computing~$\SUM_5$ composed of~two Full Adders and one Half Adder.}
	\label{figure:sumfive}
\end{figure}

Interestingly, in~some regimes, this algorithm leads to~a~circuit 
that is~\emph{provably} optimal. For example, it~adds two $n$-bit 
numbers using a~single Half Adder and $n-1$ Full Adders. The resulting
circuit has size $2+5(n-1)=5n-3$. Red'kin~\cite{Red81} proved
that there~is no~smaller circuit for this function.
At~the same time, in~various other scenarios, the algorithm leads 
to~a~suboptimal circuit. For example, the circuit from Figure~\ref{figure:sumfive} is~suboptimal as~$\SUM_5$
can be~computed by a~circuit of~size~$11$ (see~\cite{DBLP:conf/mfcs/KulikovPS22}).
In~general, whereas the algorithm produces a~circuit of~size about~$5n$
for $\SUM_5$, this function can be~computed by~a~circuit of~size about $4.5n$
as~shown by~Demenkov et~al.~\cite{DBLP:journals/ipl/DemenkovKKY10}.

In~this paper, we~generalize the construction by~Demenkov et~al.
Namely, we~show that whereas the algorithm described above produces
a~circuit of~size at~most $5n-3m$ computing $\BA$, one can construct
a~circuit of~size at~most $4.5n-m$. In~the regimes where $m$~is~small
compared to~$n$, this gives a~circuit that is~about $10\%$ smaller.
This applies to~the Dadda multiplier.
We~complement our theoretical result by~an~open source implementation
of~generators producing circuits for bit addition and multiplication.




\bibliographystyle{ACM-Reference-Format}
\bibliography{references}

\end{document}

