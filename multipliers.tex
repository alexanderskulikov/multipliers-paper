\documentclass[sigconf, review, anonymous]{acmart}

\usepackage{tikz}
\usetikzlibrary{math}

\DeclareMathOperator{\SUM}{SUM}
\DeclareMathOperator{\ADD}{ADD}
\DeclareMathOperator{\MULT}{MULT}

\begin{document}

\begin{CCSXML}
	<ccs2012>
	<concept>
	<concept_id>10010583.10010682.10010690.10010691</concept_id>
	<concept_desc>Hardware~Combinational synthesis</concept_desc>
	<concept_significance>500</concept_significance>
	</concept>
	<concept>
	<concept_id>10002950.10003624.10003625.10003628</concept_id>
	<concept_desc>Mathematics of computing~Combinatorial algorithms</concept_desc>
	<concept_significance>300</concept_significance>
	</concept>
	</ccs2012>
\end{CCSXML}

\ccsdesc[500]{Hardware~Combinational synthesis}
\ccsdesc[300]{Mathematics of computing~Combinatorial algorithms}

\keywords{multiplier, multiplication, summation, Boolean, circuit, synthesis, combinational}

\title{Smaller Circuits for Summation and Multiplication}




\begin{abstract}
	The Full Adder and Half Adder are used frequently when synthesizing Boolean circuits.
	For example, one can compute the binary representation of~the sum of~$n$~inputs bits
	using at~most~$n$ Full and Half Adders. Also, a~circuit computing the sum of~two $n$-bit
	numbers can be~composed out of~one Half Adder and $(n-1)$ Full Adders. Finally,
	to~compute the product of~two $n$-bit numbers, one first computes $n^2$ partial products
	and then adds them using about $n^2$ Full and Half Adders. Two natural strategies for
	adding the partial products were presented by~Wallace and Dadda.

	In~this paper, we~study the size of~circuits composed out of~Full and Half Adders,
	over the full binary basis. Interestingly, many such circuits are suboptimal:
	for example, already the sum of~five input bits can be~computed more efficiently
	than by~using two Full Adders and a~single Half Adder. We~show that, in~many interesting
	regimes, circuits composed out of~Full and Half Adders, can be~made about 10\% smaller.
	This applies to~Wallace and Dadda multipliers.
	We~provide open source implementation of~generators producing efficient
	circuits for summation and multiplication.
\end{abstract}


\maketitle

\section{Overview}
The Full Adder and Half Adder
(abbreviated~as FA and HA and also known~as 3:2 and 2:2 compressors, respectively)
are used frequently when synthesizing Boolean circuits. They compute $\SUM_3$
and $\SUM_2$, respectively, where the function
\[\SUM_n \colon \{0,1\}^n \to \{0,1\}^l\]
gives the binary representation of~the sum of~$n$~input bits:
\[\SUM_n(x_1, \dotsc, x_n)=(s_0, \dotsc, s_{l-1}) \text{ such that } \sum_{i=1}^{n}x_i=\sum_{i=0}^{l-1}2^is_i,\]
where $l=\lceil \log_2(n+1) \rceil$.


\begin{center}
    \begin{tikzpicture}
        %\draw[help lines] (0, -2) grid (8, 1);
        \tikzstyle{dot}=[circle, fill=black, inner sep=0mm, minimum size=1mm]
        \tikzstyle{l}=[dotted, gray, thin]

        \tikzmath{\d=0.4;}

        \begin{scope}
            \foreach \y in {-2, ..., 2}
                \node[dot] at (0, \y * \d) {};
            \node at (0, -2) {$\SUM_5$};

            \foreach \x [count=\n from 0] in {0} {
                \draw[l] (\x * \d, -1) -- (\x * \d, 1);
                \node[gray, below] at (\x * \d, -1) {$\n$};
            }
        \end{scope}

        \begin{scope}[xshift=20mm]
            \foreach \x in {-2, ..., 2} {
                \node[dot] at (\x * \d, \d / 2) {};
                \node[dot] at (\x * \d, - \d / 2) {};
            }
            \node at (0, -2) {$\ADD_5$};

            \foreach \x [count=\n from 0] in {2, 1, ..., -2} {
                \draw[l] (\x * \d, -1) -- (\x * \d, 1);
                \node[gray, below] at (\x * \d, -1) {$\n$};
            }
        \end{scope}

        \begin{scope}[xshift=55mm]
            \foreach \y in {-2, ..., 2}
                \foreach \x in {-2, ..., 2} {
                    \node[dot] at (\x * \d + \y * \d, \y * \d) {};
            }
            \node at (0, -2) {$\MULT_5$};

            \foreach \x [count=\n from 0] in {4, 3, ..., -4} {
                \draw[l] (\x * \d, -1) -- (\x * \d, 1);
                \node[gray, below] at (\x * \d, -1) {$\n$};
            }
        \end{scope}
    \end{tikzpicture}
\end{center}


\begin{center}
    \begin{tikzpicture}
        %\draw[help lines] (0, -2) grid (8, 1);
        \tikzstyle{dot}=[circle, fill=black, inner sep=0mm, minimum size=1mm]
        \tikzstyle{l}=[dotted, gray, thin]

        \tikzmath{\d=0.4;}

        \begin{scope}
            \foreach \x in {-2, ..., 2}
                \node[dot] at (\x * \d, - \d / 2) {};
            \node[dot] at (2 * \d, \d / 2) {};

            \foreach \x [count=\n from 0] in {2, 1, ..., -2} {
                \draw[l] (\x * \d, -1) -- (\x * \d, 1);
                \node[gray, below] at (\x * \d, -1) {$\n$};
            }
        \end{scope}


        \begin{scope}[xshift=30mm]
            \foreach \x in {-2, ..., 2} {
                \node[dot] at (\x * \d, \d) {};
                \node[dot] at (\x * \d, 0) {};
                \node[dot] at (\x * \d, -\d) {};
            }

            \foreach \x [count=\n from 0] in {2, 1, ..., -2} {
                \draw[l] (\x * \d, -1) -- (\x * \d, 1);
                \node[gray, below] at (\x * \d, -1) {$\n$};
            }
        \end{scope}

        \begin{scope}[xshift=60mm]
            \foreach \x/\y in {-2/-1, -1/-1, -1/0, -1/1, 1/-1, 1/0, 2/-1}
                \node[dot] at (\x * \d, \y * \d) {};

            \foreach \x [count=\n from 0] in {2, 1, ..., -2} {
                \draw[l] (\x * \d, -1) -- (\x * \d, 1);
                \node[gray, below] at (\x * \d, -1) {$\n$};
            }
        \end{scope}
    \end{tikzpicture}
\end{center}



\end{document}

