\documentclass[sigconf, review, anonymous]{acmart}

\usepackage{tikz}
\usetikzlibrary{math, fit}

\usepackage{listings}

\tikzstyle{dot}=[circle, fill=black, inner sep=0mm, minimum size=1mm]
\tikzstyle{l}=[dotted, thin]
\tikzstyle{input}=[draw=none, inner sep=.2mm]
\tikzstyle{gate}=[draw, circle, inner sep=.2mm, minimum size=2mm]
\tikzstyle{outgate}         = [gate, thick]
\tikzstyle{wire}            = [draw,->]
\tikzstyle{notwire}         = [draw,->,dashed]
\tikzstyle{pair} = [rectangle, draw, dashed, inner sep=.7mm]
\tikzmath{\d=0.4;} % distance between dots

\usepackage[textwidth=15mm]{todonotes}

\DeclareMathOperator{\SUM}{SUM}
\DeclareMathOperator{\ADD}{ADD}
\DeclareMathOperator{\MULT}{MULT}
\DeclareMathOperator{\BA}{BA}
\DeclareMathOperator{\MDFA}{MDFA}

\usepackage{newfloat}
\usepackage{listings}
\DeclareCaptionStyle{ruled}{labelfont=normalfont,labelsep=colon,strut=off} % DO NOT CHANGE THIS
\lstset{%
	basicstyle={\footnotesize\ttfamily},% footnotesize acceptable for monospace
	%numbers=left,numberstyle=\footnotesize,xleftmargin=2em,% show line numbers, remove this entire line if you don't want the numbers.
	aboveskip=0pt,belowskip=0pt,%
	showstringspaces=false,tabsize=2,breaklines=true}
\floatstyle{ruled}
\newfloat{listing}{tb}{lst}{}
\floatname{listing}{Listing}

\begin{document}

\begin{CCSXML}
	<ccs2012>
	<concept>
	<concept_id>10010583.10010682.10010690.10010691</concept_id>
	<concept_desc>Hardware~Combinational synthesis</concept_desc>
	<concept_significance>500</concept_significance>
	</concept>
	<concept>
	<concept_id>10002950.10003624.10003625.10003628</concept_id>
	<concept_desc>Mathematics of computing~Combinatorial algorithms</concept_desc>
	<concept_significance>300</concept_significance>
	</concept>
	</ccs2012>
\end{CCSXML}

\ccsdesc[500]{Hardware~Combinational synthesis}
\ccsdesc[300]{Mathematics of computing~Combinatorial algorithms}

\keywords{bit addition, summation, multiplier, multiplication, Boolean, circuit, synthesis, combinational, digital}

\title{Smaller Circuits for Bit Addition}

\begin{abstract}
    Bit addition arises virtually everywhere in~digital circuits:
    arithmetic operations,
    increment/decrement operators,
    computing addresses and table indices, and so~on.
    Since bit addition is~such a~basic task in~Boolean circuit synthesis,
    a~lot of~research has been done on~constructing efficient circuits
    for various special cases of~it. A~vast majority of~these results is~devoted to~optimizing the circuit \emph{depth} (also known as~delay).

    In~this paper, we~investigate the circuit \emph{size} (also known as~area)
    over the full binary basis of~bit addition. Though most of~the known circuits are built from Half Adders and Full Adders,
    we~show that, in~many interesting scenarios, these circuits have suboptimal size
    and can be~made about 10\% smaller. This applies to~a~wide range
    of~circuits computing various arithmetic operations.
    We~complement our theoretical result by~an~open source implementation
    of~generators producing circuits for bit addition and multiplication.
\end{abstract}



\maketitle

\section{Overview}
Bit addition arises virtually everywhere in~digital circuits:
arithmetic operations,
increment/decrement operators,
computing addresses and table indices, and so~on.
Three specific scenarios where it~is used frequently are listed below.
\begin{itemize}
	\item Adding two $n$-bit numbers.
	\item Computing a~symmetric Boolean function
		(such as~majority or~sorting).
		A~natural way of~doing this is~to~first compute
		the binary representation of~the sum of~$n$~input bits
		(that~is, to~compress $n$~bits into about $\log_2 n$ bits)
		and then to~compute the function at~hand
		out of~the computed binary representation.
	\item To~multiply two $n$-bit numbers, one may first compute
		all partial products and then sum~up the resulting bits.
\end{itemize}
In~terms of~the dot-notation introduced by~Dadda~\cite{dadda}, the three scenarios discussed above are visualized as~shown in~Figure~\ref{figure:dot1}.

\begin{figure}%[ht]
	\begin{center}
	    \begin{tikzpicture}
	        \begin{scope}[xshift=25mm]
                \foreach \x [count=\n from 0] in {0} {
                    \draw[l] (\x * \d, -1) -- (\x * \d, 1);
                    \node[below] at (\x * \d, -1) {$\n$};
                }

	            \foreach \y in {-2, ..., 2}
	                \node[dot] at (0, \y * \d) {};
	            \node at (0, -2) {$\SUM_5$};
	        \end{scope}

	        \begin{scope}[xshift=0mm]
                \foreach \x [count=\n from 0] in {2, 1, ..., -2} {
                    \draw[l] (\x * \d, -1) -- (\x * \d, 1);
                    \node[below] at (\x * \d, -1) {$\n$};
                }

	            \foreach \x in {-2, ..., 2} {
	                \node[dot] at (\x * \d, - 2 * \d) {};
	                \node[dot] at (\x * \d, - \d) {};
	            }
	            \node at (0, -2) {$\ADD_5$};
	        \end{scope}

	        \begin{scope}[xshift=55mm]
	            \foreach \x [count=\n from 0] in {4, 3, ..., -4} {
                    \draw[l] (\x * \d, -1) -- (\x * \d, 1);
                    \node[below] at (\x * \d, -1) {$\n$};
                }

	            \foreach \y in {-2, ..., 2}
	                \foreach \x in {-2, ..., 2} {
	                    \node[dot] at (\x * \d + \y * \d, \y * \d) {};
	            }
	            \node at (0, -2) {$\MULT_5$};
	        \end{scope}
	    \end{tikzpicture}
	\end{center}
	\caption{Dot diagrams for three Boolean functions: $\ADD_5$ adds two five-bit numbers, $\SUM_5$ adds five bits, and $\MULT_5$ adds five five-bit numbers.}
	\label{figure:dot1}
\end{figure}

There are many other cases where one needs to~add bits.
Say, one may want
to~add a~single bit to an~$n$-bit number (the increment operation
is a~special case),
or~to~add three $n$-bit numbers, or~to~add a~few bits of~varying significance,
see Figure~\ref{figure:dot2}.

\begin{figure}%[ht]
	\begin{center}
	    \begin{tikzpicture}
	        \begin{scope}
	            \foreach \x [count=\n from 0] in {2, 1, ..., -2} {
                    \draw[l] (\x * \d, -1) -- (\x * \d, .5);
                    \node[below] at (\x * \d, -1) {$\n$};
                }

	            \foreach \x in {-2, ..., 2}
	                \node[dot] at (\x * \d, - 2 * \d) {};
	            \node[dot] at (2 * \d, -\d) {};
	        \end{scope}


	        \begin{scope}[xshift=30mm]
	            \foreach \x [count=\n from 0] in {2, 1, ..., -2} {
                    \draw[l] (\x * \d, -1) -- (\x * \d, .5);
                    \node[below] at (\x * \d, -1) {$\n$};
                }

	            \foreach \x in {-2, ..., 2} {
	                \node[dot] at (\x * \d, 0) {};
	                \node[dot] at (\x * \d, -\d) {};
	                \node[dot] at (\x * \d, -2 * \d) {};
	            }
	        \end{scope}

	        \begin{scope}[xshift=60mm]
	            \foreach \x [count=\n from 0] in {2, 1, ..., -2} {
                    \draw[l] (\x * \d, -1) -- (\x * \d, .5);
                    \node[below] at (\x * \d, -1) {$\n$};
                }

	            \foreach \x/\y in {-2/-1, -1/-1, -1/0, -1/1, 0/-1, 1/-1, 1/0, 2/-1}
	                \node[dot] at (\x * \d, \y * \d - \d) {};
	        \end{scope}
	    \end{tikzpicture}
	\end{center}
	\caption{More scenarios of~adding bits of~varying significance.}
	\label{figure:dot2}
\end{figure}

A~function capturing all such scenarios is~known as~\emph{bit adder}
$\BA_n^{s_1, \dotsc, s_n} \colon \{0,1\}^n \to \{0,1\}^m$. It~is parameterized by~the \emph{significance vector}
$s=(s_1, \dotsc, s_n)$, takes $n$~input bits $(x_1, \dotsc, x_n)$, and outputs
the binary representation~of
\[\sum_{i=1}^{n}2^{s_i}x_i.\]
The number~$m$ of~outputs of $\BA$ is~determined by~its significance vector:
\[m=\left\lceil \log_2\left( \sum_{i=1}^{n}2^{s_i} + 1\right) \right\rceil.\]
This way, $\SUM_n=\BA^{0,0,\dotsc,0}_n$ and $\ADD_n=\BA^{0,0,1,1,\dotsc,n-1,n-1}_{2n}$.
In~dot notation, the goal of~bit addition is~to~``flatten''
the distribution of~bits, that~is, to~leave one bit at~each significance layer.
Figure~\ref{figure:baexample} gives an~example.

\begin{figure}%[ht]
	\begin{center}
		\begin{tikzpicture}
			\begin{scope}
				\foreach \x [count=\n from 0] in {2, 1, ..., -2} {
					\draw[l] (\x * \d, -1) -- (\x * \d, .5);
					\node[below] at (\x * \d, -1) {$\n$};
				}

				\foreach \x/\y in {-2/-1, -1/-1, -1/0, -1/1, 0/-1, 1/-1, 1/0, 2/-1}
				\node[dot] at (\x * \d, \y * \d - \d) {};
			\end{scope}

			\draw[dashed, ->] (1.5, -\d) -- (2.5, -\d);

			\begin{scope}[xshift=40mm]
				\foreach \x [count=\n from 0] in {3, 2, ..., -2} {
					\draw[l] (\x * \d, -1) -- (\x * \d, .5);
					\node[below] at (\x * \d, -1) {$\n$};
					\node[dot] at (\x * \d, -2 * \d) {};
				}
			\end{scope}
		\end{tikzpicture}
	\end{center}
	\caption{The function $\BA_8^{0,1,1,2,3,3,3,4} \colon \{0,1\}^7 \to \{0,1\}^6$ replaces eight bits of~significance $(0,1,1,2,3,3,3,4)$ with six bits of~significance $(0,1,2,3,4,5)$.}
	\label{figure:baexample}
\end{figure}

Since bit addition is~such a~basic task in~Boolean circuit synthesis,
a~lot of~research has been done on~constructing efficient circuits
for various special cases of~it, see, for example,
\cite{DBLP:journals/cc/PatersonZ93,
    DBLP:conf/arith/MartelORS95,
    DBLP:journals/tc/StellingMOR98,
    DBLP:conf/arith/BickerstaffSS01}.
A~vast majority of~these results is~devoted to~optimizing the circuit \emph{depth} (also known as~delay).
In~this paper, we~investigate the circuit \emph{size} (also known as~area) of~bit addition. Specifically, we~study circuits over the full binary basis.

Two basic building blocks for adding bits are known as~Half Adder~(HA)
and Full Adder~(FA). They compute the binary representation of~the sum
of~two and three bits, respectively (that~is, $\SUM_2$ and $\SUM_3$).
In~the full binary basis, they can be~implemented in~two and five gates, respectively, see Figure~\ref{figure:sum23}.

\begin{figure}%[ht]
    \begin{center}
        \begin{tikzpicture}
            \begin{scope}[label distance=-.9mm, scale=.7]
                \foreach \n/\x/\y in {1/0/1, 2/1/1}
                \node[input] (x\n) at (\x, \y) {$x_{\n}$};
                \node[gate, label=left:$w_1$] (g1) at (0,0) {$\land$};
                \node[gate, label=right:$w_0$] (g2) at (1,0) {$\oplus$};
                \foreach \f/\t in {x1/g1, x1/g2, x2/g1, x2/g2}
                \draw[->] (\f) -- (\t);

                \begin{scope}[yshift=-45mm, xshift=-5mm]
                    \foreach \n/\x/\y in {1/0/3, 2/1/3, 3/2/3}
                    \node[input] (x\n) at (\x, \y) {$x_{\n}$};
                    \node[gate,label=left:$a$] (g1) at (0.5,2) {$\oplus$};
                    \node[gate,label=left:$b$] (g2) at (1.5,2) {$\oplus$};
                    \node[gate,label=left:$c$] (g3) at (0.5,1) {$\lor$};
                    \node[gate, label=right:$w_0$] (g4) at (1.5,1) {$\oplus$};
                    \node[gate, label=right:$w_1$] (g5) at (0.5,0) {$\oplus$};
                    \foreach \f/\t in {x1/g1, x2/g1, x2/g2, x3/g2, g1/g3, g2/g3, g1/g4, g3/g5, g4/g5}
                    \draw[->] (\f) -- (\t);
                    \path (x3) edge[bend left,->] (g4);
                \end{scope}
            \end{scope}

            \begin{scope}[xshift=-40mm, yshift=10mm]
                \draw[l] (0, -1) -- (0, 0.5); \node[below] at (0, -1) {$i$};
                \node[dot, label=left:$x_1$] at (0, -0.5) {};
                \node[dot, label=left:$x_2$] at (0, 0) {};

                \draw[dashed, ->] (0.25, -.25) -- (0.75, -.25);

                \draw[l] (2, -1) -- (2, 0.5); \node[below] at (2, -1) {$i$};
                \draw[l] (1.5, -1) -- (1.5, .5); \node[below] at (1.5, -1) {$i+1$};
                \node[dot, label=left:$w_1$] at (1.5, -0.5) {};
                \node[dot, label=right:$w_0$] at (2, -0.5) {};
            \end{scope}

            \begin{scope}[xshift=-40mm, yshift=-20mm]
                \draw[l] (0, -1) -- (0, 0.5); \node[below] at (0, -1) {$i$};
                \node[dot, label=left:$x_1$] at (0, -0.75) {};
                \node[dot, label=left:$x_2$] at (0, -0.25) {};
                \node[dot, label=left:$x_3$] at (0, 0.25) {};

                \draw[dashed, ->] (0.25, -.25) -- (0.75, -.25);

                \draw[l] (2, -1) -- (2, 0.5); \node[below] at (2, -1) {$i$};
                \draw[l] (1.5, -1) -- (1.5, .5); \node[below] at (1.5, -1) {$i+1$};
                \node[dot, label=left:$w_1$] at (1.5, -0.75) {};
                \node[dot, label=right:$w_0$] at (2, -0.75) {};
            \end{scope}
        \end{tikzpicture}
    \end{center}
    \caption{The Half Adder (top) and Full Adder (bottom): dot diagrams and circuits.}
    \label{figure:sum23}
\end{figure}


Using Half Adders and Full Adders, one can synthesize a~bit adder using the following algorithm that goes back to~Napier's \emph{Rabdologiæ} (1617),
as~modernized by~Dadda~\cite{dadda}.
\begin{quote}
	Process the bits layer by~layer, in~the order of~increasing significance.
	While the current significance layer~$i$ contains at~least three bits,
	take three of~them and apply the Full Adder to~replace them with a~pair of~bits
	of~significance $i$~and~$i+1$. If~there are two bits left at~the current layer~$i$, apply the Half Adder to~them to~get a~pair of~bits of~significance $i$~and~$i+1$.
\end{quote}
This algorithm ensures that, for any vector $s \in \mathbb{Z}_{\ge 0}^n$,
\[\operatorname{size}(\BA_n^s) \le 5n-3m.\]
Indeed, each application of a~Full Adder reduces the number of~bits by~one,
hence the total cost of~all Full Adders is~at~most $5(n-m)$. A~Half Adder is~applied at~most once for each of~the significance layers, hence
the total cost of~all Half Adders is~at~most~$2m$. Hence, the total size
is~at~most $5(n-m)+2m=5n-3m$.

By~applying this algorithm to~partial products of~bits of~two input $n$-bit numbers, one gets the well-known Dadda multiplier circuit~\cite{dadda}.
For many vectors~$s$, the upper bound
$5n-3m$ is~loose:
it~does not match the size of~the actual circuit
produced by~the algorithm.
A~straightforward example is $s=(0,1,\dotsc,n-1)$:
in~this case, no~gates are needed whereas the upper bound is~$2n$.
It~is also worth noting that, in~some cases, the resulting circuit
is~\emph{provably} optimal.
For example, for the $\ADD_n$ function (that computes the sum of~two $n$-bit integers),
the method constructs a~circuit out of a~single Half Adder and $(n-1)$
Full Adders. The resulting circuit is~known as~\emph{ripple-carry adder} and has size $5n-3$.
Red'kin~\cite{Red81} proved that there~is no~smaller circuit
for this function.

At~the same time, in~many scenarios,
not only the bound $5n-3m$ is~loose,
but also the circuit produced by~the algorithm
is~suboptimal.
For example, for $\SUM_5$, it~gives a~circuit of~size~$12$ consisting
of~two Full Adders and one Half Adder, see Figure~\ref{figure:sumfive}.
However, $\SUM_5$
can be~computed by a~circuit of~size~$11$ as~shown by~\cite{DBLP:conf/mfcs/KulikovPS22} (see also Figure~\ref{figure:mdfa} later in~the text).
In~general, whereas the algorithm produces a~circuit of~size about~$5n$
for $\SUM_n$, this function can be~computed by~a~circuit of~size about $4.5n$
as~shown by~Demenkov et~al.~\cite{DBLP:journals/ipl/DemenkovKKY10}.

\begin{figure}%[ht]
	\begin{tikzpicture}
		\begin{scope}[scale=.7]
			\begin{scope}[scale=1.2, yshift=10mm]
				\begin{scope}[yshift=20mm]
					\foreach \x [count=\n from 0] in {0} {
						\draw[l] (\x * \d, -1) -- (\x * \d, 1.5);
						\node[below] at (\x * \d, -1) {$\n$};
					}
					\foreach \n in {-2,-1,...,2}
						\node[dot] at (0, \n * \d) {};

					\path (0.2, 0) edge[->, dashed] node[below] {FA} (0.8, 0);
				\end{scope}

				\begin{scope}[xshift=14mm, yshift=20mm]
					\foreach \x [count=\n from 0] in {0, -1} {
						\draw[l] (\x * \d, -1) -- (\x * \d, 1.5);
						\node[below] at (\x * \d, -1) {$\n$};
					}
					\foreach \x/\y in {0/-2, 0/-1, 0/0, -1/-2}
						\node[dot] at (\x * \d, \y * \d) {};
					\path (0.2, 0) edge[->, dashed] node[below] {FA} (0.8, 0);
				\end{scope}

				\begin{scope}[xshift=28mm, yshift=20mm]
					\foreach \x [count=\n from 0] in {0, -1} {
						\draw[l] (\x * \d, -1) -- (\x * \d, 1.5);
						\node[below] at (\x * \d, -1) {$\n$};
					}
					\foreach \x/\y in {0/-2, -1/-2, -1/-1}
						\node[dot] at (\x * \d, \y * \d) {};
					\path (0.2, 0) edge[->, dashed] node[below] {HA} (0.8, 0);
				\end{scope}

				\begin{scope}[xshift=46mm, yshift=20mm]
					\foreach \x [count=\n from 0] in {0, -1, -2} {
						\draw[l] (\x * \d, -1) -- (\x * \d, 1.5);
						\node[below] at (\x * \d, -1) {$\n$};
						\node[dot] at (\x * \d, -2 * \d) {};
					}
				\end{scope}
			\end{scope}

            \begin{scope}[yshift=-30mm, scale=.9]
                %\draw[help lines] (0, 0) grid (6, 4);
                \foreach \n/\x/\y in {x_1/0/3, x_2/1/4, x_3/2/4, x_4/4/4, x_5/5/4, w_0/6/3, w_1/6/1.5, w_2/3/0.5}
                    \node[input] (\n) at (\x, \y) {$\n$};
                \draw (0.5, 2.5) rectangle (2.5, 3.5); \node at (1.5, 3) {FA};
                \draw (3.5, 2.5) rectangle (5.5, 3.5); \node at (4.5, 3) {FA};
                \draw (0.5, 1) rectangle (5.5, 2); \node at (3, 1.5) {HA};
                \foreach \f/\t in {x_1/{0.5, 3}, x_2/{1, 3.5}, x_3/{2, 3.5}, x_4/{4, 3.5}, x_5/{5, 3.5}, {5.5, 3}/w_0, {2.5, 3}/{3.5, 3},
                {1.5, 2.5}/{1.5, 2}, {4.5, 2.5}/{4.5, 2}, {5.5, 1.5}/w_1, {3, 1}/w_2}
                    \draw[->] (\f) -- (\t);
            \end{scope}

			\begin{scope}[label distance=-1mm, xshift=70mm, yshift=20mm]
				\foreach \n/\x/\y in {1/0/3, 2/1/3, 3/2/3, 4/2.5/1, 5/3.5/1}
				\node[input] (x\n) at (\x, \y) {$x_{\n}$};
				\node[gate,label=left:$g_1$] (g1) at (0.5,2) {$\oplus$};
				\node[gate,label=left:$g_2$] (g2) at (1.5,2) {$\oplus$};
				\node[gate,label=left:$g_3$] (g3) at (0.5,1) {$\lor$};
				\node[gate,label=left:$g_4$] (g4) at (1.5,1) {$\oplus$};
				\node[gate,label=left:$g_5$] (g5) at (0.5,0) {$\oplus$};
				\node[gate,label=left:$g_6$] (g6) at (2,-1) {$\oplus$};
				\node[gate,label=right:$g_7$] (g7) at (3,-1) {$\oplus$};
				\node[gate,label=right:$g_8$] (g8) at (2,-2) {$\lor$};
				\node[gate, label=right:$w_0$] (g9) at (3,-2) {$\oplus$};
				\node[gate, label=right:$g_9$] (g10) at (2,-3) {$\oplus$};
				\node[gate, label=right:$w_1$] (g11) at (2,-4) {$\oplus$};
				\node[gate, label=left:$w_2$] (g12) at (1,-4) {$\land$};

				\foreach \f/\t in {x1/g1, x2/g1, x2/g2, x3/g2, g1/g3, g2/g3, g1/g4, g3/g5, g4/g5, g4/g6, x4/g6, x4/g7, x5/g7, g6/g8, g7/g8, g8/g10, g6/g9, g9/g10, g10/g11, g10/g12}
				\draw[->] (\f) -- (\t);

				\path (x3) edge[->,bend left] (g4);
				\path (x5) edge[->,bend left=35] (g9);
				\path (g5) edge[->,bend right=25] (g11);
				\path (g5) edge[->,bend right=15] (g12);

				\draw[dashed] (-0.5,-0.25) rectangle (2,2.5);
				\draw[dashed] (1.25,-3.25) rectangle (4,-0.5);
				\draw[dashed] (0,-3.5) rectangle (3,-4.5);
			\end{scope}
		\end{scope}
	\end{tikzpicture}
	\caption{A~circuit of~size~$12$ computing~$\SUM_5$ composed of~two Full Adders and one Half Adder: dot notation (top left), block structure (bottom left), and a~circuit (right).}
	\label{figure:sumfive}
\end{figure}


In~this paper, we~generalize the construction by~Demenkov et~al.
Namely, we~prove an~upper bound $4.5n-2m$
for the circuit size of~bit addition.
In~the regimes where $m$~is~small
compared to~$n$, this gives a~circuit that is~about $10\%$ smaller.
This applies to~the Dadda multiplier.
We~complement our theoretical result by~an~open source implementation
of~generators producing circuits for bit addition and multiplication.

\section{General Setting}
In~this section,
we~formally introduce the Boolean functions
studied in~this paper as~well as~the main building blocks
for computing them.

\subsection{Boolean Functions}
The main Boolean function studied in~this paper
is~\emph{bit adder}
\[\BA_n^{s_1, \dotsc, s_n} \colon \{0,1\}^n \to \{0,1\}^m.\]
It~computes the binary representation of~the weighted sum of~input bits:
\[\BA(x_1, \dotsc, x_n)=(y_0, \dotsc, y_{m-1}) \colon \sum_{i=1}^{n}2^{s_i}x_i=\sum_{i=0}^{m-1}2^iy_i.\]
The number~$m$ of~outputs of $\BA_n^s$ is~determined by~its significance vector $s=(s_1, \dotsc, s_n)$:
\[m=\left\lceil \log_2\left( \sum_{i=1}^{n}2^{s_i} + 1\right) \right\rceil.\]
Many practically important Boolean functions can be~computed using bit summation.
\begin{itemize}
    \item The function $\SUM_n \colon \{0,1\}^n \to \{0,1\}^{\lceil \log_2(n+1) \rceil}$
    computes the sum of~$n$ bits: \[\SUM_n(x_1, \dotsc, x_n)=\ADD_n^{0,0,\dotsc,0}(x_1, \dotsc, x_n).\]
	\item The function $\ADD_n \colon \{0,1\}^{2n} \to \{0,1\}^{n+1}$ computes the sum
	of~two $n$-bit numbers:
    \begin{multline*}
        \ADD_n(x_0, \dotsc, x_{n-1}, y_0, \dotsc, y_{n-1})\\
        =\BA_{2n}^{0,\dotsc,n-1,0,\dotsc,n-1}(x_0, \dotsc, x_{n-1}, y_0, \dotsc, y_{n-1}).
    \end{multline*}
	\item The function $\MULT_n \colon \{0,1\}^{2n} \to \{0,1\}^{2n}$ computes the product
	of~two $n$-bit numbers:
	\begin{multline*}
	\MULT_n(x_0, \dotsc, x_{n-1}, y_0, \dotsc, y_{n-1})\\=\BA_{n^2}^{(i+j)_{0 \le i, j < n}}\left(\left(x_i \land y_j\right)_{0 \le i, j < n}\right).
	\end{multline*}
\end{itemize}

\todo[inline]{mention that we consider non-degenerate cases (when $m \le n$) only}


\subsection{Boolean Circuits}
A~circuit is~a~natural way of~computing Boolean functions.
It~is an~acyclic directed graph of in-degree $0$~and~$2$ whose $n+2$~source
nodes are labeled with input variables
$x_1, \dotsc, x_n$ and constants $0$~and~$1$, whereas all other nodes
are labeled with binary Boolean operations.
The inputs nodes are called input gates, all other nodes are called internal gates.
Each gate computes
a~(single-output) Boolean function of~$x_1, \dotsc, x_n$. If~$m$ gates of the
circuit are marked as~outputs, it~computes a~function of~the form $\{0,1\}^n \to \{0,1\}^m$.
The size of a~circuit is~its number of~internal gates.

\subsection{Basic Building Blocks}
As~discussed before, the Half Adder and Full Adder are basic building
blocks for computing bit addition. Figure~\ref{figure:sum17fa}
shows how to~synthesize a~circuit of~size~$66$ computing $\SUM_{17}$
out of~three Half Adders and twelve Full Adders.
It~is not difficult to~see that a~similar block structure can
be~used for any~$n$ yielding a~circuit of size at~most~$5n$ for $\SUM_n$.

\begin{figure}
    \begin{center}
        \begin{tikzpicture}[scale=0.47]
            %\draw[help lines] (2, 0) grid (17, 6);
            \foreach \n in {2,...,17} {
                \tikzmath{\j=int(\n - 1);}
                \node[input] (\n) at (\n, 6) {$x_{\j}$};
            }
            \foreach \n in {2, 4, ..., 16} {
                \draw (\n - 0.25, 4.5) rectangle (\n + 1.25, 5.5);
                \node at (\n + 0.5, 5) {\ifnumless{\n}{3}{HA}{FA}};
                \tikzmath{\i=int(\n + 1);}
                \draw[->] (\n) -- (\n, 5.5);
                \draw[->] (\i) -- (\i, 5.5);
                \ifnumless{\n}{16}{\draw[->] (\n + 1.25, 5) -- (\n + 1.75, 5);}{}
            }
            \foreach \n in {2, 6, 10, 14} {
                \draw (\n - 0.25, 3) rectangle (\n + 3.25, 4);
                \node at (\n + 1.5, 3.5) {\ifnumless{\n}{3}{HA}{FA}};
                \draw[->] (\n + 0.5, 4.5) -- (\n + 0.5, 4);
                \draw[->] (\n + 2.5, 4.5) -- (\n + 2.5, 4);
                \ifnumless{\n}{14}{\draw[->] (\n + 3.25, 3.5) -- (\n + 3.75, 3.5);}{}
            }
            \draw (1.75, 1.5) rectangle (9.25, 2.5); \node at (5.5, 2) {HA};
            \draw (9.75, 1.5) rectangle (17.25, 2.5); \node at (13.5, 2) {FA};
            \draw (1.75, 0) rectangle (17.25, 1); \node at (9.5, 0.5) {HA};

            \foreach \n/\x/\y in {w_0/18/5, w_1/18/3.5, w_2/18/2, w_3/18/0.5, w_4/9.5/-0.75}
                \node[input] (\n) at (\x, \y) {$\n$};

            \foreach \f/\t in {{17.25, 5}/w_0, {17.25, 3.5}/w_1, {17.25, 2}/w_2, {17.25, 0.5}/w_3, {3.5, 3}/{3.5, 2.5},
            {7.5, 3}/{7.5, 2.5}, {11.5, 3}/{11.5, 2.5}, {15.5, 3}/{15.5, 2.5},
            {5.5, 1.5}/{5.5, 1}, {13.5, 1.5}/{13.5, 1}, {9.5, 0}/w_4, {9.25, 2}/{9.75, 2}}
                \draw[->] (\f) -- (\t);
        \end{tikzpicture}
    \end{center}
    \caption{A~circuit computing $\SUM_{16}$ composed out~of four Half Adders and eleven Full Adders. Its size is $4 \cdot 2 + 11 \cdot 5=63$.}
    \label{figure:sum17fa}
\end{figure}

It~turns out that better circuit designs are possible for $\SUM_n$
as~shown by~Demenkov et~al.~\cite{DBLP:journals/ipl/DemenkovKKY10}.
Consider two consecutive Full Adders shown on~the top left of~Figure~\ref{figure:mdfa}. The corresponding function DFA (for Double Full Adder) has the following specification: \[\operatorname{DFA}(x_1, x_2,\dotsc,x_5)=(b_0,b_1,a_1) \colon x_1+\dotsb+x_5=b_0+2(b_1+a_1).\]
Then, MDFA (for Modified Double Full Adder) has the following specification:
\[\MDFA(x_1 \oplus x_2, x_2, x_3, x_4, x_4 \oplus x_5)=(b_0, a_1, a_1 \oplus b_1).\]
That~is, for pairs of~bits $(x_1, x_2)$, $(x_4, x_5)$, and $(a_1, b_1)$
it~uses a~slightly different encoding: $(p, p \oplus q)$ instead of~$(p,q)$.
It~allows one to~compute MDFA in~eight gates (whereas the circuit size of~DFA is~10). Moreover, the corresponding circuit of~size~eight is~just a~part
of~an~optimal circuit of~size~$11$ computing~$\SUM_5$ shown on~the right
of~Figure~\ref{figure:mdfa}.

\begin{figure}%[ht]
    \begin{center}
        \begin{tikzpicture}[scale=.7]
            \begin{scope}[scale=.7]
                %\draw[help lines] (0,0) grid (10,6);
                \draw (1,0) rectangle (3,2); \node at (2,1) {FA};
                \draw (5,0) rectangle (7,2); \node at (6,1) {FA};
                \foreach \n/\x/\y in {3/0/1, 2/1.5/3, 1/2.5/3, 4/5.5/3, 5/6.5/3}
                \node[input] (\n) at (\x,\y) {$x_{\n}$};
                \foreach \n/\t/\x/\y in {a1/a_1/2/-1, b1/b_1/6/-1, b0/b_0/8/1}
                \node[input] (\n) at (\x,\y) {$\t$};
                \draw[->] (3)--(1,1);
                \draw[->] (2)--(1.5,2);
                \draw[->] (1)--(2.5,2);
                \draw[->] (4)--(5.5,2);
                \draw[->] (5)--(6.5,2);
                \draw[->] (3,1)--(5,1);
                \draw[->] (7,1)--(b0);
                \draw[->] (2,0)--(a1);
                \draw[->] (6,0)--(b1);
            \end{scope}

            \begin{scope}[scale=.7,yshift=-60mm]
                %\draw[help lines] (0,0) grid (10,6);
                \draw (1,0) rectangle (7,2); \node at (4,1) {MDFA};
                \foreach \n/\x/\y in {3/0/1, 2/1.5/4, 1/2.5/4, 4/5.5/4, 5/6.5/4}
                \node[input] (\n) at (\x,\y) {$x_{\n}$};
                \node[gate] (xor1) at (2.5,3) {$\oplus$};
                \node[gate] (xor2) at (6.5,3) {$\oplus$};
                \foreach \n/\t/\x/\y in {a1/a_1/2/-1, b1/{a_1 \oplus b_1}/6/-1, b0/b_0/8/1}
                \node[input] (\n) at (\x,\y) {$\t$};
                \draw[->] (3)--(1,1);
                \draw[->] (2)--(1.5,2);
                \draw[->] (1) -- (xor1); \draw[->] (2) -- (xor1);
                \draw[->] (xor1) -- (2.5,2);
                \draw[->] (4)--(5.5,2);
                \draw[->] (5)-- (xor2); \draw[->] (xor2) -- (6.5,2); \draw[->] (4) -- (xor2);
                %\draw[->] (3,1)--(5,1);
                \draw[->] (7,1)--(b0);
                \draw[->] (2,0)--(a1);
                \draw[->] (6,0)--(b1);
            \end{scope}

            \begin{scope}[yscale=.8, xshift=70mm, yshift=-20mm]
                %\draw[help lines] (0,-3) grid (4,4);
                \draw[draw=none, rounded corners=0,fill=gray!20] (0,1.5)--(1,1.5)--(1,2.5)--(2,2.5)--(2,0.5)--(2.5,0.5)--(2.5,-0.5)--(3.5,-0.5)--
                (3.5,-2.5)--(0,-2.5)--(0,1.5);


                \foreach \n/\x/\y in {1/0/3, 2/1/3, 3/2/3, 4/2.5/1, 5/3.5/1}
                \node[input] (x\n) at (\x, \y) {$x_{\n}$};
                \node[gate] (g1) at (0.5,2) {$\oplus$};
                \node[gate] (g2) at (1.5,2) {$\oplus$};
                \node[gate] (g3) at (0.5,1) {$\lor$};
                \node[gate] (g4) at (1.5,1) {$\oplus$};
                \node[gate, label=left:$a_1$] (g5) at (0.5,0) {$\oplus$};
                \node[gate] (g6) at (2,0) {$\oplus$};
                \node[gate] (g7) at (3,0) {$\oplus$};
                \node[gate] (g8) at (2,-1) {$>$};
                \node[gate, label=right:$b_0$] (g9) at (3,-1) {$\oplus$};
                \node[gate, label=right:$a_1 \oplus b_1$] (g10) at (2,-2) {$\oplus$};
                \node[gate] (g11) at (1.5,-3) {$>$};

                \foreach \f/\t in {x1/g1, x2/g1, x2/g2, x3/g2, g1/g3, g2/g3, g1/g4, g3/g5, g4/g5, x4/g6, g4/g6, x4/g7, x5/g7, g6/g8, g7/g8, g7/g9, g3/g10, g8/g10, g10/g11, g5/g11}
                \draw[->] (\f) -- (\t);

                \path (x3) edge[->,bend left] (g4);
                \path (g4) edge[->,bend left=20] (g9);
            \end{scope}
        \end{tikzpicture}
    \end{center}
    \caption{Two consecutive Full Adders (top left), the MDFA block (bottom left), and an~optimal circuit for $\SUM_5$ (right) whose highlighted part computes~MDFA.}
    \label{figure:mdfa}
\end{figure}

Using MDFA blocks, one can compute $\SUM_n$ roughly as~follows:
\begin{enumerate}
    \item Compute $x_2 \oplus x_3, x_4 \oplus x_5, \dotsc, x_{n-1} \oplus x_n$ ($n/2$ gates).
    \item Apply at~most~$n/2$ $\MDFA$ blocks (no~more than $4n$~gates).
    \item The last MDFA block outputs two bits: $a$~and~$a\oplus b$. Instead of~them, one needs to~compute $a \oplus b$ and $a \land b$. To~achieve this,
    it~suffices to apply $x>y=(x \land \overline{y})$ operation:
    \(a \land b = a>(a \oplus b)\).
\end{enumerate}
This gives an~upper bound $4.5n$ for $\SUM_n$, its formal proof can
be~found in~\cite{DBLP:journals/ipl/DemenkovKKY10}. Figure~\ref{figure:sum17mdfa} gives an~example of~the corresponding design
for~$\SUM_{17}$.

\begin{figure}%[!ht]
	\begin{center}
		\begin{tikzpicture}[scale=0.47]
			\foreach \n in {2,...,17} {
                \tikzmath{\j=int(\n-1);}
			    \node[input] (\n) at (\n,6) {$x_{\j}$};
            }
			\foreach \i in {1,...,8} {
				\tikzmath{\k=int(2*\i); \j=int(2*\i+1);}
				\node[gate] (a) at (\j,5) {$\oplus$};
				\draw[->] (\k) -- (a); \draw[->] (\j) -- (a);
				\draw[->] (a) -- (\j,4); \draw[->] (\k) -- (\k,4);
			}
			\foreach \x/\y/\w/\t in {2/4/3/MDFA', 6/4/3/MDFA, 10/4/3/MDFA, 14/4/3/MDFA, 2/2.5/7/MDFA', 10/2.5/7/MDFA, 2/1/15/MDFA'} {
				\draw (\x-0.15,\y) rectangle (\x+\w+0.15,\y-1);
				\node at (\x+\w/2,\y-0.5) {\t};
			}

    		\node[input] (c) at (18,3.5) {$w_0$};
			\draw[->] (17.15,3.5) -- (c);

            \node[input] (w_1) at (18, 2) {$w_1$}; \draw[->] (17.15, 2) -- (w_1);
            \node[input] (w_2) at (18, 0.5) {$w_2$}; \draw[->] (17.15, 0.5) -- (w_2);

            \foreach \x in {3, 4, 7, 8, 11, 12, 15, 16}
			     \draw[->] (\x,3) -- (\x,2.5);
			\foreach \x in {4, 7, 12, 15}
			     \draw[->] (\x,1.5) -- (\x,1);
			\foreach \x/\y in {5/3.5, 9/3.5, 13/3.5, 9/2}
			     \draw[->] (\x+0.15,\y) -- (\x+0.85,\y);

			\node[input] (w3) at (12,-2) {$w_3$};
			\draw[->] (12,0) -- (w3);
			\node[gate] (x) at (7,-1) {$>$};
			\draw[->] (7,0) -- (x);
			\node[input] (w4) at (7,-2) {$w_4$};
			\draw[->] (x) -- (w4);
			\path (12,0) edge[->,out=-135,in=0] (x);
		\end{tikzpicture}
	\end{center}
	\caption{A~circuit computing $\SUM_{16}$ composed out~of eight $\oplus$-gates at~the top, three MDFA' blocks, four MDFA blocks, and one final gate. Its size is $8+3 \cdot 6 + 4 \cdot 8 + 1=59$. The block MDFA' results from MDFA by~assigning zero to~one of~its inputs ($x_3$ in~terms of~Figure~\ref{figure:mdfa}); since this input feeds two gates, MDFA' can be~implemented using six gates.}
	\label{figure:sum17mdfa}
\end{figure}

\section{New Upper Bound for Circuit Size of~Bit Addition}
In~this section, we~prove a~new upper bound $4.5-2m$ for the circuit size
of~bit addition. For regimes where $m$~is small compared to~$n$, this~is
better than $5n-3m$ by~about $10\%$. This applies to~$\MULT_n$ and $\SUM_n$.
\begin{theorem}
    For any vector $s \in \mathbb{Z}_{\ge 0}^n$,
    \[\operatorname{size}(\BA_n^s) \le 4.5n-2m.\]
\end{theorem}

In~the proof, we~use the following straightforward observation:
if~the significance vector~$s$ contains a~single zero (say, $s_1=0$), then computing $\BA^s_n$ is~the same as~computing $\BA^{s'}_{n-1}$, where $s'=(s_2-1,\dotsc, s_n-1)$. We~refer to~this operation as~\emph{shifting}.
See Figure~\ref{figure:shifting} for an~example.

\begin{figure}
    \begin{center}
        \begin{tikzpicture}
            \begin{scope}
                \foreach \x [count=\n from 0] in {2, 1, ..., -2} {
                    \draw[l] (\x * \d, -1) -- (\x * \d, .5);
                    \node[below] at (\x * \d, -1) {$\n$};
                }

                \foreach \x/\y in {-2/-1, -1/-1, -1/0, -1/1, 0/-1, 1/-1, 1/0, 2/-1}
                \node[dot] at (\x * \d, \y * \d - \d) {};
            \end{scope}

            \draw[dashed, ->] (1.5, -\d) -- (3, -\d);

            \begin{scope}[xshift=40mm]
                \foreach \x [count=\n from 0] in {2, ..., -1} {
                    \draw[l] (\x * \d, -1) -- (\x * \d, .5);
                    \node[below] at (\x * \d, -1) {$\n$};
                }

                \foreach \x/\y in {-2/-1, -1/-1, -1/0, -1/1, 0/-1, 1/-1, 1/0}
                \node[dot] at (\x * \d + \d, \y * \d - \d) {};
            \end{scope}
        \end{tikzpicture}
    \end{center}
    \caption{Shifting: computing $\BA_8^{0,1,1,2,3,3,3,4}$ is~the same as~computing $\BA_7^{0,0,1,2,2,2,3}$.}
    \label{figure:shifting}
\end{figure}

\begin{proof}
    As~the first step, we~do the following: at~every significance layer,
    we~break all bits, except for possibly one, into pairs and compute
    the parity for every pair. This takes at~most $n/2$ gates.

    \begin{center}
        \begin{tikzpicture}
                \foreach \x [count=\n from 0] in {2, 1, ..., -2} {
                    \draw[l] (\x * \d, -1) -- (\x * \d, 1);
                    \node[below] at (\x * \d, -1) {$\n$};
                }

                \foreach [count=\n] \x/\y in {-2/-1, -2/0, -2/1, -2/2, -2/3, -1/-1, -1/0, -1/1, 0/-1, 1/-1, 1/0, 1/1, 1/2, 2/-1, 2/0}
                    \node[dot] (\n) at (\x * \d, \y * \d - \d) {};

                \foreach \i/\j in {1/2, 3/4, 6/7, 10/11, 12/13, 14/15}
                    \node[fit=(\i) (\j), pair] {};
        \end{tikzpicture}
    \end{center}

    Then, it~remains to~prove that one can compute the sum of~$n$ bits
    using $4n-2m$ gates if~every significance layer contains at~most one bit
    without a~pair. We~prove this by~induction on~$n$. The base case $n=1$ is~clear: in~this case, the circuit size is~zero (nothing needs to~be summed~up) and the upper bound is~at~least zero since we~only consider
    non-degenerate case where $m \le n$. To~prove the induction step,
    denote by~$l$ the number of~bits on~the least significance layer and consider the following cases.

    \begin{enumerate}
        \item $l=1$. In~this case, we~just shift.
        By~the induction hypothesis, the rest can be~computed by~a~circuit
        of~size at~most
        \[4(n-1)-2(m-1)=4n-2m-2<4n-2m.\]

        \item $l=2$. Then, the corresponding two bits~$x_1$ and~~$x_2$ are paired meaning that their sum $x_1 \oplus x_2$ is~computed already.
        Then, we~compute their carry
        \[c=x_1 > (x_1 \oplus x_2)=x_1 \land x_2\]
        and transfer~it to~the next layer.
        If~this layer has an~unpaired bit~$b$, we~pair $b$~and~$c$
        by~computing $b \oplus c$. Finally, we~shift.
        By~the induction hypothesis, the size of~the resulting circuit is~at~most
        \[1+1+4(n-1)-2(m-1)=4n-2m.\]

        \begin{center}
            \begin{tikzpicture}
                \begin{scope}
                    \foreach \x [count=\n from 0] in {2, 1} {
                        \draw[l] (\x * \d, -1) -- (\x * \d, 0.5);
                        \node[below] at (\x * \d, -1) {$\n$};
                    }

                    \foreach [count=\n] \x/\y in {2/-1, 2/0, 1/-1, 1/0, 1/1}
                        \node[dot] (\n) at (\x * \d, \y * \d - \d) {};

                    \foreach \i/\j in {1/2, 3/4}
                        \node[fit=(\i) (\j), pair] {};
                \end{scope}

                \begin{scope}[xshift=20mm]
                    \foreach \x [count=\n from 0] in {2, 1} {
                        \draw[l] (\x * \d, -1) -- (\x * \d, 0.5);
                        \node[below] at (\x * \d, -1) {$\n$};
                    }

                    \foreach [count=\n] \x/\y in {2/-1, 1/-1, 1/0, 1/1, 1/2}
                    \node[dot] (\n) at (\x * \d, \y * \d - \d) {};

                    \foreach \i/\j in {2/3}
                        \node[fit=(\i) (\j), pair] {};
                \end{scope}

                \begin{scope}[xshift=40mm]
                    \foreach \x [count=\n from 0] in {2, 1} {
                        \draw[l] (\x * \d, -1) -- (\x * \d, 0.5);
                        \node[below] at (\x * \d, -1) {$\n$};
                    }

                    \foreach [count=\n] \x/\y in {2/-1, 1/-1, 1/0, 1/1, 1/2}
                    \node[dot] (\n) at (\x * \d, \y * \d - \d) {};

                    \foreach \i/\j in {2/3, 4/5}
                        \node[fit=(\i) (\j), pair] {};
                \end{scope}

                \begin{scope}[xshift=60mm]
                    \foreach \x [count=\n from 0] in {2} {
                        \draw[l] (\x * \d, -1) -- (\x * \d, 0.5);
                        \node[below] at (\x * \d, -1) {$\n$};
                    }

                    \foreach [count=\n] \x/\y in {2/-1, 2/0, 2/1, 2/2}
                        \node[dot] (\n) at (\x * \d, \y * \d - \d) {};

                    \foreach \i/\j in {1/2, 3/4}
                    \node[fit=(\i) (\j), pair] {};
                \end{scope}

                %\draw[help lines] (0, -2) grid (7, 1);
                \path (1.2, -0.5) edge[->, dashed] node[below] {$\land$} (2, -0.5);
                \path (3.2, -0.5) edge[->, dashed] node[below] {$\oplus$} (4, -0.5);
                \path (5.2, -0.5) edge[->, dashed] node[below] {shift} (6, -0.5);
            \end{tikzpicture}
        \end{center}

        \item $l=3$. For the corresponding three bits $x_1,x_2,x_3$,
        we~have $x_1 \oplus x_2$, $x_2$, and $x_3$ (that~is, $x_1$~and~$x_2$ are paired). We~apply the Full Adder to~the three bits. This costs four gates, as~$x_1 \oplus x_2$ is~already computed and $x_1$ is~not needed
        (recall Figure~\ref{figure:sum23}). The sum bit stays on~the same layer, whereas the carry bit~$c$ goes to~the next layer. Then, we~pair~$c$
        with an~unpaired bit on~the next layer if~needed and shift. This gives an~upper bound
        \[4+1+4(n-2)-2(m-1)<4n-2m.\]

        \begin{center}
            \begin{tikzpicture}
                \begin{scope}
                    \foreach \x [count=\n from 0] in {2, 1} {
                        \draw[l] (\x * \d, -1) -- (\x * \d, 0.5);
                        \node[below] at (\x * \d, -1) {$\n$};
                    }

                    \foreach [count=\n] \x/\y in {2/-1, 2/0, 2/1, 1/-1, 1/0, 1/1}
                        \node[dot] (\n) at (\x * \d, \y * \d - \d) {};

                    \foreach \i/\j in {1/2, 4/5}
                        \node[fit=(\i) (\j), pair] {};
                \end{scope}

                \begin{scope}[xshift=20mm]
                    \foreach \x [count=\n from 0] in {2, 1} {
                        \draw[l] (\x * \d, -1) -- (\x * \d, 0.5);
                        \node[below] at (\x * \d, -1) {$\n$};
                    }

                    \foreach [count=\n] \x/\y in {2/-1, 1/-1, 1/0, 1/1, 1/2}
                    \node[dot] (\n) at (\x * \d, \y * \d - \d) {};

                    \foreach \i/\j in {2/3}
                    \node[fit=(\i) (\j), pair] {};
                \end{scope}

                \begin{scope}[xshift=40mm]
                    \foreach \x [count=\n from 0] in {2, 1} {
                        \draw[l] (\x * \d, -1) -- (\x * \d, 0.5);
                        \node[below] at (\x * \d, -1) {$\n$};
                    }

                    \foreach [count=\n] \x/\y in {2/-1, 1/-1, 1/0, 1/1, 1/2}
                    \node[dot] (\n) at (\x * \d, \y * \d - \d) {};

                    \foreach \i/\j in {2/3, 4/5}
                    \node[fit=(\i) (\j), pair] {};
                \end{scope}

                \begin{scope}[xshift=60mm]
                    \foreach \x [count=\n from 0] in {2} {
                        \draw[l] (\x * \d, -1) -- (\x * \d, 0.5);
                        \node[below] at (\x * \d, -1) {$\n$};
                    }

                    \foreach [count=\n] \x/\y in {2/-1, 2/0, 2/1, 2/2}
                    \node[dot] (\n) at (\x * \d, \y * \d - \d) {};

                    \foreach \i/\j in {1/2, 3/4}
                    \node[fit=(\i) (\j), pair] {};
                \end{scope}

                %\draw[help lines] (0, -2) grid (7, 1);
                \path (1.2, -0.5) edge[->, dashed] node[below] {FA} (2, -0.5);
                \path (3.2, -0.5) edge[->, dashed] node[below] {$\oplus$} (4, -0.5);
                \path (5.2, -0.5) edge[->, dashed] node[below] {shift} (6, -0.5);
            \end{tikzpicture}
        \end{center}

        \item $l=4k$. Apply MDFA' to~two pairs to~produce an~unpaired bit.
        For the remaining $2k-2$ pairs, keep applying MDFA, each time reusing the unpaired bit. Then, we~shift. The upper bound~is
        \[6+8(k-1)+4(n-2k)-2(m-1)=4n-2m.\]

        \begin{center}
            \begin{tikzpicture}
                \begin{scope}
                    \foreach \x [count=\n from 0] in {2, 1} {
                        \draw[l] (\x * \d, -1) -- (\x * \d, 2.5);
                        \node[below] at (\x * \d, -1) {$\n$};
                    }

                    \foreach [count=\n] \x/\y in {2/-1, 2/0, 2/1, 2/2, 2/3, 2/4, 2/5, 2/6, 1/-1, 1/0}
                        \node[dot] (\n) at (\x * \d, \y * \d - \d) {};

                    \foreach \i/\j in {1/2, 3/4, 5/6, 7/8, 9/10}
                        \node[fit=(\i) (\j), pair] {};
                \end{scope}

                \begin{scope}[xshift=20mm]
                    \foreach \x [count=\n from 0] in {2, 1} {
                        \draw[l] (\x * \d, -1) -- (\x * \d, 2.5);
                        \node[below] at (\x * \d, -1) {$\n$};
                    }

                    \foreach [count=\n] \x/\y in {2/-1, 2/0, 2/1, 2/2, 1/-1, 1/0, 1/1, 1/2, 2/3}
                    \node[dot] (\n) at (\x * \d, \y * \d - \d) {};

                    \foreach \i/\j in {1/2, 3/4, 5/6, 7/8}
                        \node[fit=(\i) (\j), pair] {};
                \end{scope}

                \begin{scope}[xshift=40mm]
                    \foreach \x [count=\n from 0] in {2, 1} {
                        \draw[l] (\x * \d, -1) -- (\x * \d, 2.5);
                        \node[below] at (\x * \d, -1) {$\n$};
                    }

                    \foreach [count=\n] \x/\y in {1/-1, 1/0, 1/1, 1/2, 1/3, 1/4, 2/-1}
                    \node[dot] (\n) at (\x * \d, \y * \d - \d) {};

                    \foreach \i/\j in {1/2, 3/4, 5/6}
                    \node[fit=(\i) (\j), pair] {};
                \end{scope}

                \begin{scope}[xshift=60mm]
                    \foreach \x [count=\n from 0] in {2} {
                        \draw[l] (\x * \d, -1) -- (\x * \d, 2.5);
                        \node[below] at (\x * \d, -1) {$\n$};
                    }

                    \foreach [count=\n] \x/\y in {2/-1, 2/0, 2/1, 2/2, 2/3, 2/4}
                    \node[dot] (\n) at (\x * \d, \y * \d - \d) {};

                    \foreach \i/\j in {1/2, 3/4, 5/6}
                        \node[fit=(\i) (\j), pair] {};
                \end{scope}

                %\draw[help lines] (0, -2) grid (7, 1);
                \path (1.2, -0.5) edge[->, dashed] node[below] {MDFA'} (2, -0.5);
                \path (3.2, -0.5) edge[->, dashed] node[below, text width=20mm, align=center] {MDFA\\ $k-1$} (4, -0.5);
                \path (5.2, -0.5) edge[->, dashed] node[below] {shift} (6, -0.5);
            \end{tikzpicture}
        \end{center}

        \item $l=4k+1$. Apply MDFA $k$~times, then shift. The upper bound~is
        \[6+8(k-1)+4(n-2k)-2(m-1)=4n-2m.\]

        \begin{center}
            \begin{tikzpicture}
                \begin{scope}
                    \foreach \x [count=\n from 0] in {2, 1} {
                        \draw[l] (\x * \d, -1) -- (\x * \d, 2.5);
                        \node[below] at (\x * \d, -1) {$\n$};
                    }

                    \foreach [count=\n] \x/\y in {2/-1, 2/0, 2/1, 2/2, 2/3, 2/4, 2/5, 2/6, 2/7, 1/-1, 1/0}
                    \node[dot] (\n) at (\x * \d, \y * \d - \d) {};

                    \foreach \i/\j in {1/2, 3/4, 5/6, 7/8, 10/11}
                    \node[fit=(\i) (\j), pair] {};
                \end{scope}

                \begin{scope}[xshift=20mm]
                    \foreach \x [count=\n from 0] in {2, 1} {
                        \draw[l] (\x * \d, -1) -- (\x * \d, 2.5);
                        \node[below] at (\x * \d, -1) {$\n$};
                    }

                    \foreach [count=\n] \x/\y in {2/-1, 2/0, 2/1, 2/2, 2/3, 2/4, 1/-1}
                    \node[dot] (\n) at (\x * \d, \y * \d - \d) {};

                    \foreach \i/\j in {1/2, 3/4, 5/6}
                    \node[fit=(\i) (\j), pair] {};
                \end{scope}

                \begin{scope}[xshift=40mm]
                    \foreach \x [count=\n from 0] in {2} {
                        \draw[l] (\x * \d, -1) -- (\x * \d, 2.5);
                        \node[below] at (\x * \d, -1) {$\n$};
                    }

                    \foreach [count=\n] \x/\y in {2/-1, 2/0, 2/1, 2/2, 2/3, 2/4}
                    \node[dot] (\n) at (\x * \d, \y * \d - \d) {};

                    \foreach \i/\j in {1/2, 3/4, 5/6}
                    \node[fit=(\i) (\j), pair] {};
                \end{scope}

                %\draw[help lines] (0, -2) grid (7, 1);
                \path (1.2, -0.5) edge[->, dashed] node[below, text width=20mm, align=center] {MDFA\\$k$} (2, -0.5);
                \path (3.2, -0.5) edge[->, dashed] node[below] {shift} (4, -0.5);
            \end{tikzpicture}
        \end{center}

        \item $l=4k+2$. Compute an~$\land$ of~two bits from the same pair:
        this leaves their sum at~the current layer and puts the just computed
        carry bit to~the next layer. If~needed, compute the parity of~an~unpaired
        bit with the just transferred carry bit. Then, apply MDFA $k$~times and shift.
        Overall, the upper bound~is
        \[1+1+8k+4(n-2k-1)-2(m-1)=4n-2m.\]

        \begin{center}
            \begin{tikzpicture}
                \begin{scope}
                    \foreach \x [count=\n from 0] in {2, 1} {
                        \draw[l] (\x * \d, -1) -- (\x * \d, 1.5);
                        \node[below] at (\x * \d, -1) {$\n$};
                    }

                    \foreach [count=\n] \x/\y in {2/-1, 2/0, 2/1, 2/2, 2/3, 2/4, 1/-1, 1/0, 1/1}
                    \node[dot] (\n) at (\x * \d, \y * \d - \d) {};

                    \foreach \i/\j in {1/2, 3/4, 5/6, 7/8}
                    \node[fit=(\i) (\j), pair] {};
                \end{scope}

                \begin{scope}[xshift=20mm]
                    \foreach \x [count=\n from 0] in {2, 1} {
                        \draw[l] (\x * \d, -1) -- (\x * \d, 1.5);
                        \node[below] at (\x * \d, -1) {$\n$};
                    }

                    \foreach [count=\n] \x/\y in {2/-1, 2/0, 2/1, 2/2, 1/-1, 1/0, 1/1, 1/2, 2/3}
                    \node[dot] (\n) at (\x * \d, \y * \d - \d) {};

                    \foreach \i/\j in {1/2, 3/4, 5/6, 7/8}
                    \node[fit=(\i) (\j), pair] {};
                \end{scope}

                \begin{scope}[xshift=40mm]
                    \foreach \x [count=\n from 0] in {2, 1} {
                        \draw[l] (\x * \d, -1) -- (\x * \d, 1.5);
                        \node[below] at (\x * \d, -1) {$\n$};
                    }

                    \foreach [count=\n] \x/\y in {1/-1, 1/0, 1/1, 1/2, 1/3, 1/4, 2/-1}
                    \node[dot] (\n) at (\x * \d, \y * \d - \d) {};

                    \foreach \i/\j in {1/2, 3/4, 5/6}
                    \node[fit=(\i) (\j), pair] {};
                \end{scope}

                \begin{scope}[xshift=60mm]
                    \foreach \x [count=\n from 0] in {2} {
                        \draw[l] (\x * \d, -1) -- (\x * \d, 1.5);
                        \node[below] at (\x * \d, -1) {$\n$};
                    }

                    \foreach [count=\n] \x/\y in {2/-1, 2/0, 2/1, 2/2, 2/3, 2/4}
                    \node[dot] (\n) at (\x * \d, \y * \d - \d) {};

                    \foreach \i/\j in {1/2, 3/4, 5/6}
                    \node[fit=(\i) (\j), pair] {};
                \end{scope}

                %\draw[help lines] (0, -2) grid (7, 1);
                \path (1.2, -0.5) edge[->, dashed] node[below] {$\land$, $\oplus$} (2, -0.5);
                \path (3.2, -0.5) edge[->, dashed] node[below, text width=20mm, align=center] {MDFA\\ $k$} (4, -0.5);
                \path (5.2, -0.5) edge[->, dashed] node[below] {shift} (6, -0.5);
            \end{tikzpicture}
        \end{center}

        \item $l=4k+3$. Apply the Full Adder to~a~pair of~bits and the unpaired bit.
        If~needed, pair the just transferred carry bit with an~unpaired bit from
        the next layer. Then, apply MDFA $k$~times and shift. The resulting upper
        bound~is
        \[4+1+8k+4(n-2k-2)-2(m-1)<4n-2m.\]

        \begin{center}
            \begin{tikzpicture}
                \begin{scope}
                    \foreach \x [count=\n from 0] in {2, 1} {
                        \draw[l] (\x * \d, -1) -- (\x * \d, 2);
                        \node[below] at (\x * \d, -1) {$\n$};
                    }

                    \foreach [count=\n] \x/\y in {2/-1, 2/0, 2/1, 2/2, 2/3, 2/4, 1/-1, 1/0, 1/1, 2/5}
                        \node[dot] (\n) at (\x * \d, \y * \d - \d) {};

                    \foreach \i/\j in {1/2, 3/4, 5/6, 7/8}
                        \node[fit=(\i) (\j), pair] {};
                \end{scope}

                \begin{scope}[xshift=20mm]
                    \foreach \x [count=\n from 0] in {2, 1} {
                        \draw[l] (\x * \d, -1) -- (\x * \d, 2);
                        \node[below] at (\x * \d, -1) {$\n$};
                    }

                    \foreach [count=\n] \x/\y in {2/-1, 2/0, 2/1, 2/2, 1/-1, 1/0, 1/1, 1/2, 2/3}
                    \node[dot] (\n) at (\x * \d, \y * \d - \d) {};

                    \foreach \i/\j in {1/2, 3/4, 5/6, 7/8}
                    \node[fit=(\i) (\j), pair] {};
                \end{scope}

                \begin{scope}[xshift=40mm]
                    \foreach \x [count=\n from 0] in {2, 1} {
                        \draw[l] (\x * \d, -1) -- (\x * \d, 2);
                        \node[below] at (\x * \d, -1) {$\n$};
                    }

                    \foreach [count=\n] \x/\y in {1/-1, 1/0, 1/1, 1/2, 1/3, 1/4, 2/-1}
                    \node[dot] (\n) at (\x * \d, \y * \d - \d) {};

                    \foreach \i/\j in {1/2, 3/4, 5/6}
                    \node[fit=(\i) (\j), pair] {};
                \end{scope}

                \begin{scope}[xshift=60mm]
                    \foreach \x [count=\n from 0] in {2} {
                        \draw[l] (\x * \d, -1) -- (\x * \d, 2);
                        \node[below] at (\x * \d, -1) {$\n$};
                    }

                    \foreach [count=\n] \x/\y in {2/-1, 2/0, 2/1, 2/2, 2/3, 2/4}
                    \node[dot] (\n) at (\x * \d, \y * \d - \d) {};

                    \foreach \i/\j in {1/2, 3/4, 5/6}
                    \node[fit=(\i) (\j), pair] {};
                \end{scope}

                %\draw[help lines] (0, -2) grid (7, 1);
                \path (1.2, -0.5) edge[->, dashed] node[below] {FA, $\oplus$} (2, -0.5);
                \path (3.2, -0.5) edge[->, dashed] node[below, text width=20mm, align=center] {MDFA\\ $k$} (4, -0.5);
                \path (5.2, -0.5) edge[->, dashed] node[below] {shift} (6, -0.5);
            \end{tikzpicture}
        \end{center}
    \end{enumerate}
\end{proof}


%
%\begin{proof}
%Before we move to the prove exactly, let's think what constant \(d\) we can choose for
%    \[\operatorname{size}(\BA_n^s) \le 4.5n-md?\]
%
%First, consider a simple example to obtain initial bounds for the factor \( d \). Suppose we have $\BA^{0,0,1,2\dotsc,n - 1}_{n + 1}$. Such a scenario yields \( m = n + 1 \). The number of required input gates is at least \( 2n \), because we must use at least Half Adder on each layer. Hence we have:
%\[
%4.5n - d m \geq 2n \Rightarrow d \leq 2.5
%\]
%
%To analyze further, consider some specific useful scenarios. First, adding two \( n \)-bit numbers directly is convenient since one may use Full Adders at each layer. This scenario provides an upper bound close to 4, which is already less than 4.5.
%
%We computed the factor \( d \) experimentally using the following formula:
%\[
%d = \frac{4.5 \cdot n \cdot m + n - \text{len(ckt.gates)}}{\text{len(res)}}
%\]
%
%The obtained results for \( n = 100 \) and varying \( m \) were:
%
%\[
%\begin{array}{c|cccccccccc}
%m & 3 & 4 & 5 & 6 & 7 & 8 & 9 & 10 & 100 & 1000 \\
%\hline
%d & 4.45 & 4.36 & 4.42 & 3.95 & 4.10 & 4.30 & 4.38 & 4.10 & 3.86 & 3.81
%\end{array}
%\]
%
%This indicates that \( d \) is generally a convenient factor for estimation (That give us hope to prove \(d = 2.5\) some day).
%
%Next, we consider the savings from using special building blocks (e.g., MDFA). Each block internally has 8 gates, effectively reducing the number of bits from 5 to 3. Using these blocks alone would yield a complexity of about \(4n\). But we need to get pairs of \(x, x \oplus y \) somehow, we can achieve this efficiency, reducing the complexity slightly further to \(4.5n\) from initially \(5n\).
%
%We will move from least significant layer and count how many gates we spend. From previous layer we can receive some numbers of pairs \(x, x \oplus y \) and \textbf{No more than one single bit} (namely carry in out notation).
%From this moment on we will pay many attention on each case. A careful analysis of all possible scenarios at each layer is required. These scenarios include:
%\begin{enumerate}
%    \item A carry exists, and the number of new bits is even.
%    \item A carry exists, and the number of new bits is odd.
%    \item No carry, and an even number of new bits.
%    \item No carry, and an odd number of new bits.
%\end{enumerate}
%
%The most challenging scenarios occur when there's an odd number of bits at the current level with existing carry bit.
%
%
%\textbf{1. Carry exists, even numbers of new bits:}\\
%Denote the carry bit as \( z \). We pair remaining pair of bits with efficiently. If an odd number of pairs occurs, one pair from the previous level and one pair from the current level will yield 5 gates for 2 bits, hence \( d' = 2.5 \). If even pairs are present, then clearly we moved from 5 bits to only 2 in the last MDFA that give us \( d' = 4.5 - 1 = 3.5 \).
%
%\textbf{2. Carry exists, odd numbers of new bits:}\\
%Critically, at the next level, we pass no more than one unpaired bit. This ensures one-bit reduction per level, achieving \( d' = 3 \). Accounting for the carry bit \( z \), the complexity remains controlled (at most 4), maintaining our target bounds.
%
%\textbf{3. No carry, odd numbers of new bits:}\\
%Similar reasoning as above. Again, pairing is imperfect, yielding \( d' = 2.5 \). This scenario aligns closely with the earlier bounds.
%
%\textbf{4. No carry, even numbers of new bits (critical scenario):}\\
%We have an even number of bits, but we deliberately leave one pair aside for handling separately due to the carry-less structure. Processing the leftover pair requires at most 8 gates, thus achieving \( d' = 2.5 \).
%
%In this case, we must employ a special MDFA block (with a constant 0 on input and two pairs with  \(x, x \oplus y \) type), ensuring efficient handling. This block involves an XOR gate for carry adjustments, using 7 gates to handle 2 bits, leading precisely to:
%\[
%d' = 4.5 - 3.5 = 1
%\]
%
%\todo[inline]{I need to work more carefully with all these cases. Now it looks like: trivial, easy to check, hence... \\
%But also I don't want to make it weird way with paragraphs for each mini-block}
%
%\textbf{Result:}\\
%All scenarios thus remain within our bounds \(d' \ge 1\), concluding our proof that the total size of the constructed circuit does not exceed \(4.5n - m\).
%\end{proof}

\section{Implementation and Experimental Evaluation}
We~implemented efficient generators of~our new circuits in~the Cirbo framework~\cite{DBLP:journals/corr/abs-2412-14933}. Listing~\ref{listing:first} shows
that how to~generate an~efficient circuit computing $\SUM_{31}$ in~just two lines of~code.

\begin{listing}
	\caption{TBW}
	\label{listing:first}
	\begin{lstlisting}
from cirbo.core.circuit import Circuit
from cirbo.synthesis.generation.arithmetics.summation import add_sum_n_bits

ckt = Circuit.bare_circuit(input_size=31)
add_sum_n_bits(ckt, ckt.inputs)
ckt.save_to_file('sum31.bench')
	\end{lstlisting}
\end{listing}

\begin{table}
	\begin{center}
		\small
		\begin{tabular}{lrrrrrrrrrr}
			\toprule
			$n$ & 7 & 15 & 31 & 63 & 127 & 255 & 511 & 1023 & 2047 \\
			\midrule
			FA & 20 & 55 & 130 & 285 & 600 & 1235 & 2510 & 5065 & 10180 \\
			MDFA & 19 & 51 & 119 & 259 & 543 & 1115 & 2263 & 4563 & 9167 \\
			& 5\%  & 7.3\%  & 8.5\%  & 9.1\%  & 9.5\%  & 9.7\%  & 9.8\%  & 9.9\%  & 10\% \\
			\bottomrule
		\end{tabular}
	\end{center}
	\caption{Comparing the size of~circuits for $\SUM_n$ composed out of~Full Adders with
	circuits composed out of~MDFA. The bottom row shows the improvement in~percents.}
	\label{table:first}
\end{table}


\section{Conclusion and Further Directions}

In this paper, we~presented smaller circuits for bit addition.
In~many practically relevant scenarios, the described circuits
are about 10\% smaller than the known circuits composed
out~of Half Adders and Full Adders.
Also, we~implemented generators that allow one
to~produce the corresponding circuits using a~single line of~code
via the \texttt{Cirbo} package~\cite{DBLP:journals/corr/abs-2412-14933}.

The next natural step would~be to~optimize the depth of~the presented circuits.
Whereas the depth of~our circuits is~linear,
we~have been interested in~the circuit size mainly and
have not tried to~optimize the depth. We~believe that it~should be~possible
to~adjust our circuits to~make the depth logarithmic. It~would be~interesting
to~analyze this formally and to~answer the following questions:
\begin{enumerate}
	\item Is~it possible to~compute $\BA_n^s$ in~depth $O(\log n)$, for any $s \in \mathbb{Z}_{\ge 0}^n$?
	\item How small can the constant hidden in~$O(\log n)$~be?
	\item Is~it possible to~optimize both depth and size simultaneously?
\end{enumerate}


\bibliographystyle{ACM-Reference-Format}
\bibliography{references}

\end{document}

